\documentclass[]{article}
\usepackage{lmodern}
\usepackage{amssymb,amsmath}
\usepackage{ifxetex,ifluatex}
\usepackage{fixltx2e} % provides \textsubscript
\ifnum 0\ifxetex 1\fi\ifluatex 1\fi=0 % if pdftex
  \usepackage[T1]{fontenc}
  \usepackage[utf8]{inputenc}
\else % if luatex or xelatex
  \ifxetex
    \usepackage{mathspec}
  \else
    \usepackage{fontspec}
  \fi
  \defaultfontfeatures{Ligatures=TeX,Scale=MatchLowercase}
\fi
% use upquote if available, for straight quotes in verbatim environments
\IfFileExists{upquote.sty}{\usepackage{upquote}}{}
% use microtype if available
\IfFileExists{microtype.sty}{%
\usepackage{microtype}
\UseMicrotypeSet[protrusion]{basicmath} % disable protrusion for tt fonts
}{}
\usepackage[margin=1in]{geometry}
\usepackage{hyperref}
\hypersetup{unicode=true,
            pdftitle={DATA 605 - Discussion 1},
            pdfauthor={Joshua Sturm},
            pdfborder={0 0 0},
            breaklinks=true}
\urlstyle{same}  % don't use monospace font for urls
\usepackage{color}
\usepackage{fancyvrb}
\newcommand{\VerbBar}{|}
\newcommand{\VERB}{\Verb[commandchars=\\\{\}]}
\DefineVerbatimEnvironment{Highlighting}{Verbatim}{commandchars=\\\{\}}
% Add ',fontsize=\small' for more characters per line
\usepackage{framed}
\definecolor{shadecolor}{RGB}{248,248,248}
\newenvironment{Shaded}{\begin{snugshade}}{\end{snugshade}}
\newcommand{\KeywordTok}[1]{\textcolor[rgb]{0.13,0.29,0.53}{\textbf{#1}}}
\newcommand{\DataTypeTok}[1]{\textcolor[rgb]{0.13,0.29,0.53}{#1}}
\newcommand{\DecValTok}[1]{\textcolor[rgb]{0.00,0.00,0.81}{#1}}
\newcommand{\BaseNTok}[1]{\textcolor[rgb]{0.00,0.00,0.81}{#1}}
\newcommand{\FloatTok}[1]{\textcolor[rgb]{0.00,0.00,0.81}{#1}}
\newcommand{\ConstantTok}[1]{\textcolor[rgb]{0.00,0.00,0.00}{#1}}
\newcommand{\CharTok}[1]{\textcolor[rgb]{0.31,0.60,0.02}{#1}}
\newcommand{\SpecialCharTok}[1]{\textcolor[rgb]{0.00,0.00,0.00}{#1}}
\newcommand{\StringTok}[1]{\textcolor[rgb]{0.31,0.60,0.02}{#1}}
\newcommand{\VerbatimStringTok}[1]{\textcolor[rgb]{0.31,0.60,0.02}{#1}}
\newcommand{\SpecialStringTok}[1]{\textcolor[rgb]{0.31,0.60,0.02}{#1}}
\newcommand{\ImportTok}[1]{#1}
\newcommand{\CommentTok}[1]{\textcolor[rgb]{0.56,0.35,0.01}{\textit{#1}}}
\newcommand{\DocumentationTok}[1]{\textcolor[rgb]{0.56,0.35,0.01}{\textbf{\textit{#1}}}}
\newcommand{\AnnotationTok}[1]{\textcolor[rgb]{0.56,0.35,0.01}{\textbf{\textit{#1}}}}
\newcommand{\CommentVarTok}[1]{\textcolor[rgb]{0.56,0.35,0.01}{\textbf{\textit{#1}}}}
\newcommand{\OtherTok}[1]{\textcolor[rgb]{0.56,0.35,0.01}{#1}}
\newcommand{\FunctionTok}[1]{\textcolor[rgb]{0.00,0.00,0.00}{#1}}
\newcommand{\VariableTok}[1]{\textcolor[rgb]{0.00,0.00,0.00}{#1}}
\newcommand{\ControlFlowTok}[1]{\textcolor[rgb]{0.13,0.29,0.53}{\textbf{#1}}}
\newcommand{\OperatorTok}[1]{\textcolor[rgb]{0.81,0.36,0.00}{\textbf{#1}}}
\newcommand{\BuiltInTok}[1]{#1}
\newcommand{\ExtensionTok}[1]{#1}
\newcommand{\PreprocessorTok}[1]{\textcolor[rgb]{0.56,0.35,0.01}{\textit{#1}}}
\newcommand{\AttributeTok}[1]{\textcolor[rgb]{0.77,0.63,0.00}{#1}}
\newcommand{\RegionMarkerTok}[1]{#1}
\newcommand{\InformationTok}[1]{\textcolor[rgb]{0.56,0.35,0.01}{\textbf{\textit{#1}}}}
\newcommand{\WarningTok}[1]{\textcolor[rgb]{0.56,0.35,0.01}{\textbf{\textit{#1}}}}
\newcommand{\AlertTok}[1]{\textcolor[rgb]{0.94,0.16,0.16}{#1}}
\newcommand{\ErrorTok}[1]{\textcolor[rgb]{0.64,0.00,0.00}{\textbf{#1}}}
\newcommand{\NormalTok}[1]{#1}
\usepackage{graphicx,grffile}
\makeatletter
\def\maxwidth{\ifdim\Gin@nat@width>\linewidth\linewidth\else\Gin@nat@width\fi}
\def\maxheight{\ifdim\Gin@nat@height>\textheight\textheight\else\Gin@nat@height\fi}
\makeatother
% Scale images if necessary, so that they will not overflow the page
% margins by default, and it is still possible to overwrite the defaults
% using explicit options in \includegraphics[width, height, ...]{}
\setkeys{Gin}{width=\maxwidth,height=\maxheight,keepaspectratio}
\IfFileExists{parskip.sty}{%
\usepackage{parskip}
}{% else
\setlength{\parindent}{0pt}
\setlength{\parskip}{6pt plus 2pt minus 1pt}
}
\setlength{\emergencystretch}{3em}  % prevent overfull lines
\providecommand{\tightlist}{%
  \setlength{\itemsep}{0pt}\setlength{\parskip}{0pt}}
\setcounter{secnumdepth}{0}
% Redefines (sub)paragraphs to behave more like sections
\ifx\paragraph\undefined\else
\let\oldparagraph\paragraph
\renewcommand{\paragraph}[1]{\oldparagraph{#1}\mbox{}}
\fi
\ifx\subparagraph\undefined\else
\let\oldsubparagraph\subparagraph
\renewcommand{\subparagraph}[1]{\oldsubparagraph{#1}\mbox{}}
\fi

%%% Use protect on footnotes to avoid problems with footnotes in titles
\let\rmarkdownfootnote\footnote%
\def\footnote{\protect\rmarkdownfootnote}

%%% Change title format to be more compact
\usepackage{titling}

% Create subtitle command for use in maketitle
\newcommand{\subtitle}[1]{
  \posttitle{
    \begin{center}\large#1\end{center}
    }
}

\setlength{\droptitle}{-2em}
  \title{DATA 605 - Discussion 1}
  \pretitle{\vspace{\droptitle}\centering\huge}
  \posttitle{\par}
  \author{Joshua Sturm}
  \preauthor{\centering\large\emph}
  \postauthor{\par}
  \predate{\centering\large\emph}
  \postdate{\par}
  \date{01/31/2018}


\begin{document}
\maketitle

Page 43, problem C31.

Row-reduce the matrix without the aid of a calculator, indicating the
row operations you are using at each step using the notation of
Definition RO.

I'll use Gauss-Jordan elimination.

\[\begin{bmatrix}
1 & 2 & -4 \\
-3 & -1 & -3 \\
-2 & 1 & -7
\end{bmatrix}\]

\[R_2 = R_2 + 3R_1, \ \ R_3 = R_3 + 2R_1
\begin{bmatrix}
1 & 2 & -4 \\
0 & 5 & -15 \\
0 & 5 & -15
\end{bmatrix}\]

\[R_2 = R_2\times \frac{1}{5} \
\begin{bmatrix}
1 & 2 & -4 \\
0 & 1 & -3 \\
0 & 5 & -15
\end{bmatrix}\]

\[R_3 = R_3 - 5R_2
\begin{bmatrix}
1 & 2 & -4 \\
0 & 1 & -3 \\
0 & 0 & 0
\end{bmatrix}\]

\[R_1 = R_1 -2R_2
\begin{bmatrix}
1 & 0 & 2 \\
0 & 1 & -3 \\
0 & 0 & 0
\end{bmatrix}\]

The matrix is now in reduced row echelon form.

To verify, we can perform the operations in R.

\begin{Shaded}
\begin{Highlighting}[]
\CommentTok{# Set up our matrix}
\NormalTok{a <-}\StringTok{ }\KeywordTok{matrix}\NormalTok{(}\KeywordTok{c}\NormalTok{(}\DecValTok{1}\NormalTok{,}\DecValTok{2}\NormalTok{,}\OperatorTok{-}\DecValTok{4}\NormalTok{,}\OperatorTok{-}\DecValTok{3}\NormalTok{,}\OperatorTok{-}\DecValTok{1}\NormalTok{,}\OperatorTok{-}\DecValTok{3}\NormalTok{,}\OperatorTok{-}\DecValTok{2}\NormalTok{,}\DecValTok{1}\NormalTok{,}\OperatorTok{-}\DecValTok{7}\NormalTok{), }\DataTypeTok{ncol=}\DecValTok{3}\NormalTok{, }\DataTypeTok{nrow=}\DecValTok{3}\NormalTok{, }\DataTypeTok{byrow=}\NormalTok{T)}
\NormalTok{a}
\end{Highlighting}
\end{Shaded}

\begin{verbatim}
##      [,1] [,2] [,3]
## [1,]    1    2   -4
## [2,]   -3   -1   -3
## [3,]   -2    1   -7
\end{verbatim}

I found this row-reducing function on github
(\url{https://gist.github.com/ZPears/0583aae73aa06d8abd9e}).

\begin{Shaded}
\begin{Highlighting}[]
\NormalTok{convertToRREF <-}\StringTok{ }\ControlFlowTok{function}\NormalTok{(matrix) \{}

\NormalTok{  m <-}\StringTok{ }\KeywordTok{nrow}\NormalTok{(matrix)}
\NormalTok{  n <-}\StringTok{ }\KeywordTok{ncol}\NormalTok{(matrix)}
\NormalTok{  currCol <-}\StringTok{ }\DecValTok{1}
\NormalTok{  nonZeroRowCount <-}\StringTok{ }\DecValTok{0}
  
  \ControlFlowTok{while}\NormalTok{ ( (currCol }\OperatorTok{<}\StringTok{ }\NormalTok{n}\OperatorTok{+}\DecValTok{1}\NormalTok{) }\OperatorTok{&}\StringTok{ }\NormalTok{(nonZeroRowCount}\OperatorTok{+}\DecValTok{1} \OperatorTok{<=}\StringTok{ }\NormalTok{m) )  \{}

    \ControlFlowTok{if}\NormalTok{ (}\KeywordTok{sum}\NormalTok{(matrix[(nonZeroRowCount}\OperatorTok{+}\DecValTok{1}\NormalTok{)}\OperatorTok{:}\NormalTok{m, currCol]) }\OperatorTok{==}\StringTok{ }\DecValTok{0}\NormalTok{) \{}
      
\NormalTok{      currCol <-}\StringTok{ }\NormalTok{currCol }\OperatorTok{+}\StringTok{ }\DecValTok{1}

\NormalTok{    \} }\ControlFlowTok{else}\NormalTok{ \{}
      
\NormalTok{      rowIndex <-}\StringTok{ }\DecValTok{0}
    
      \ControlFlowTok{for}\NormalTok{ (i }\ControlFlowTok{in}\NormalTok{ nonZeroRowCount}\OperatorTok{+}\DecValTok{1}\OperatorTok{:}\NormalTok{m) \{}
        
        \ControlFlowTok{if}\NormalTok{ (matrix[i,currCol] }\OperatorTok{!=}\StringTok{ }\DecValTok{0}\NormalTok{) \{}
          
\NormalTok{          rowIndex <-}\StringTok{ }\NormalTok{i}
          \ControlFlowTok{break}
          
\NormalTok{        \}}
        
\NormalTok{      \}}
      
\NormalTok{      nonZeroRowCount <-}\StringTok{ }\NormalTok{nonZeroRowCount }\OperatorTok{+}\StringTok{ }\DecValTok{1}
      
      \CommentTok{# switch rows}
      
\NormalTok{      row1 <-}\StringTok{ }\NormalTok{matrix[rowIndex,]}
\NormalTok{      row2 <-}\StringTok{ }\NormalTok{matrix[nonZeroRowCount,]}
\NormalTok{      matrix[rowIndex,] <-}\StringTok{ }\NormalTok{row2}
\NormalTok{      matrix[nonZeroRowCount,] <-}\StringTok{ }\NormalTok{row1}
      
      \CommentTok{# Use the second row operation (mult. by scalar) }
      \CommentTok{# to convert the entry in row }
      \CommentTok{# nonZeroRowCount and column currCol to a 1.}
      
\NormalTok{      matrix[nonZeroRowCount,] <-}\StringTok{ }\NormalTok{(}\DecValTok{1}\OperatorTok{/}\NormalTok{matrix[nonZeroRowCount,currCol]) }\OperatorTok{*}\StringTok{ }\NormalTok{matrix[nonZeroRowCount,]}

      \CommentTok{# Use the third row operation with row nonZeroRowCount}
      \CommentTok{# to convert every other entry of column j to zero.}
      
      \ControlFlowTok{for}\NormalTok{ (k }\ControlFlowTok{in} \DecValTok{1}\OperatorTok{:}\NormalTok{m) \{}
        
        \ControlFlowTok{if}\NormalTok{ ( (matrix[k, currCol] }\OperatorTok{!=}\StringTok{ }\DecValTok{0}\NormalTok{) }\OperatorTok{&}\StringTok{ }\NormalTok{(k }\OperatorTok{!=}\StringTok{ }\NormalTok{nonZeroRowCount) ) \{}
          
          \CommentTok{# use row r (nonZeroRowCount) and row op 3 to make A[k, currCol] = 0}
          
\NormalTok{          scalar <-}\StringTok{ }\NormalTok{matrix[k, currCol] }\OperatorTok{/}\StringTok{ }\NormalTok{matrix[nonZeroRowCount, currCol] }
          
\NormalTok{          matrix[k, ] <-}\StringTok{ }\OperatorTok{-}\DecValTok{1} \OperatorTok{*}\StringTok{ }\NormalTok{scalar }\OperatorTok{*}\StringTok{ }\NormalTok{matrix[nonZeroRowCount, ] }\OperatorTok{+}\StringTok{ }\NormalTok{matrix[k, ]}

\NormalTok{        \}}
        
\NormalTok{      \}}
      
      \CommentTok{# increment and repeat}
      
\NormalTok{      currCol <-}\StringTok{ }\NormalTok{currCol }\OperatorTok{+}\StringTok{ }\DecValTok{1}
      
\NormalTok{    \}}
        
\NormalTok{  \}}
  
  \KeywordTok{return}\NormalTok{(matrix)}
      
\NormalTok{\}}
\end{Highlighting}
\end{Shaded}

Finally, let's verify that we get the same answer.

\begin{Shaded}
\begin{Highlighting}[]
\KeywordTok{convertToRREF}\NormalTok{(a)}
\end{Highlighting}
\end{Shaded}

\begin{verbatim}
##      [,1] [,2] [,3]
## [1,]    1    0    2
## [2,]    0    1   -3
## [3,]    0    0    0
\end{verbatim}


\end{document}
