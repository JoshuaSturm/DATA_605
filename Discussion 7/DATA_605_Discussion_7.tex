\documentclass[]{article}
\usepackage{lmodern}
\usepackage{amssymb,amsmath}
\usepackage{ifxetex,ifluatex}
\usepackage{fixltx2e} % provides \textsubscript
\ifnum 0\ifxetex 1\fi\ifluatex 1\fi=0 % if pdftex
  \usepackage[T1]{fontenc}
  \usepackage[utf8]{inputenc}
\else % if luatex or xelatex
  \ifxetex
    \usepackage{mathspec}
  \else
    \usepackage{fontspec}
  \fi
  \defaultfontfeatures{Ligatures=TeX,Scale=MatchLowercase}
\fi
% use upquote if available, for straight quotes in verbatim environments
\IfFileExists{upquote.sty}{\usepackage{upquote}}{}
% use microtype if available
\IfFileExists{microtype.sty}{%
\usepackage{microtype}
\UseMicrotypeSet[protrusion]{basicmath} % disable protrusion for tt fonts
}{}
\usepackage[margin=1in]{geometry}
\usepackage{hyperref}
\hypersetup{unicode=true,
            pdftitle={DATA 605 - Discussion 7},
            pdfauthor={Joshua Sturm},
            pdfborder={0 0 0},
            breaklinks=true}
\urlstyle{same}  % don't use monospace font for urls
\usepackage{graphicx,grffile}
\makeatletter
\def\maxwidth{\ifdim\Gin@nat@width>\linewidth\linewidth\else\Gin@nat@width\fi}
\def\maxheight{\ifdim\Gin@nat@height>\textheight\textheight\else\Gin@nat@height\fi}
\makeatother
% Scale images if necessary, so that they will not overflow the page
% margins by default, and it is still possible to overwrite the defaults
% using explicit options in \includegraphics[width, height, ...]{}
\setkeys{Gin}{width=\maxwidth,height=\maxheight,keepaspectratio}
\IfFileExists{parskip.sty}{%
\usepackage{parskip}
}{% else
\setlength{\parindent}{0pt}
\setlength{\parskip}{6pt plus 2pt minus 1pt}
}
\setlength{\emergencystretch}{3em}  % prevent overfull lines
\providecommand{\tightlist}{%
  \setlength{\itemsep}{0pt}\setlength{\parskip}{0pt}}
\setcounter{secnumdepth}{0}
% Redefines (sub)paragraphs to behave more like sections
\ifx\paragraph\undefined\else
\let\oldparagraph\paragraph
\renewcommand{\paragraph}[1]{\oldparagraph{#1}\mbox{}}
\fi
\ifx\subparagraph\undefined\else
\let\oldsubparagraph\subparagraph
\renewcommand{\subparagraph}[1]{\oldsubparagraph{#1}\mbox{}}
\fi

%%% Use protect on footnotes to avoid problems with footnotes in titles
\let\rmarkdownfootnote\footnote%
\def\footnote{\protect\rmarkdownfootnote}

%%% Change title format to be more compact
\usepackage{titling}

% Create subtitle command for use in maketitle
\newcommand{\subtitle}[1]{
  \posttitle{
    \begin{center}\large#1\end{center}
    }
}

\setlength{\droptitle}{-2em}
  \title{DATA 605 - Discussion 7}
  \pretitle{\vspace{\droptitle}\centering\huge}
  \posttitle{\par}
  \author{Joshua Sturm}
  \preauthor{\centering\large\emph}
  \postauthor{\par}
  \predate{\centering\large\emph}
  \postdate{\par}
  \date{03/13/2018}


\begin{document}
\maketitle

Section 5.1, Exercise 7, Page 197

\section{\texorpdfstring{7. A die is rolled until the first time \(T\)
that a six turns
up}{7. A die is rolled until the first time T that a six turns up}}\label{a-die-is-rolled-until-the-first-time-t-that-a-six-turns-up}

\subsection{\texorpdfstring{(a) What is the probability distribution for
\(T\)?}{(a) What is the probability distribution for T?}}\label{a-what-is-the-probability-distribution-for-t}

\subsubsection{Solution}\label{solution}

Since we're looking for the first success, \(T\) is a geometric random
variable, with a distribution (density) of
\(P(T = n) = p\cdot(1-p)^{n-1} \ \ \ \ \ n = 1, 2, 3, 4, ...\)

\(P(T = n) = (\frac{1}{6})\cdot(\frac{5}{6})^{n-1} \ \ \ \ \ n = 1, 2, 3, 4, ...\).

\subsection{\texorpdfstring{(b) Find
\(P(T > 3)\)}{(b) Find P(T \textgreater{} 3)}}\label{b-find-pt-3}

\subsubsection{Solution}\label{solution-1}

From Example 5.1, we find that
\(P(T > 3) = (1-p)^{3} = (\frac{5}{6})^{3} = \frac{125}{216} \approx\)
0.58.

\subsection{\texorpdfstring{(c) Find
\(P(T > 6|T > 3)\)}{(c) Find P(T \textgreater{} 6\textbar{}T \textgreater{} 3)}}\label{c-find-pt-6t-3}

\subsubsection{Solution}\label{solution-2}

Also from Example 5.1, we find that
\(P(T > 6| T > 3) = \frac{P(T > 6)}{P(T > 3)} = \frac{\frac{5}{6}^{6}}{\frac{5}{6}^{3}} = (\frac{5}{6})^{3} = \frac{125}{216} \approx\)
0.58.


\end{document}
