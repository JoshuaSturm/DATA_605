\documentclass[]{article}
\usepackage{lmodern}
\usepackage{amssymb,amsmath}
\usepackage{ifxetex,ifluatex}
\usepackage{fixltx2e} % provides \textsubscript
\ifnum 0\ifxetex 1\fi\ifluatex 1\fi=0 % if pdftex
  \usepackage[T1]{fontenc}
  \usepackage[utf8]{inputenc}
\else % if luatex or xelatex
  \ifxetex
    \usepackage{mathspec}
  \else
    \usepackage{fontspec}
  \fi
  \defaultfontfeatures{Ligatures=TeX,Scale=MatchLowercase}
\fi
% use upquote if available, for straight quotes in verbatim environments
\IfFileExists{upquote.sty}{\usepackage{upquote}}{}
% use microtype if available
\IfFileExists{microtype.sty}{%
\usepackage{microtype}
\UseMicrotypeSet[protrusion]{basicmath} % disable protrusion for tt fonts
}{}
\usepackage[margin=1in]{geometry}
\usepackage{hyperref}
\hypersetup{unicode=true,
            pdftitle={DATA 605 - Discussion 6},
            pdfauthor={Joshua Sturm},
            pdfborder={0 0 0},
            breaklinks=true}
\urlstyle{same}  % don't use monospace font for urls
\usepackage{graphicx,grffile}
\makeatletter
\def\maxwidth{\ifdim\Gin@nat@width>\linewidth\linewidth\else\Gin@nat@width\fi}
\def\maxheight{\ifdim\Gin@nat@height>\textheight\textheight\else\Gin@nat@height\fi}
\makeatother
% Scale images if necessary, so that they will not overflow the page
% margins by default, and it is still possible to overwrite the defaults
% using explicit options in \includegraphics[width, height, ...]{}
\setkeys{Gin}{width=\maxwidth,height=\maxheight,keepaspectratio}
\IfFileExists{parskip.sty}{%
\usepackage{parskip}
}{% else
\setlength{\parindent}{0pt}
\setlength{\parskip}{6pt plus 2pt minus 1pt}
}
\setlength{\emergencystretch}{3em}  % prevent overfull lines
\providecommand{\tightlist}{%
  \setlength{\itemsep}{0pt}\setlength{\parskip}{0pt}}
\setcounter{secnumdepth}{0}
% Redefines (sub)paragraphs to behave more like sections
\ifx\paragraph\undefined\else
\let\oldparagraph\paragraph
\renewcommand{\paragraph}[1]{\oldparagraph{#1}\mbox{}}
\fi
\ifx\subparagraph\undefined\else
\let\oldsubparagraph\subparagraph
\renewcommand{\subparagraph}[1]{\oldsubparagraph{#1}\mbox{}}
\fi

%%% Use protect on footnotes to avoid problems with footnotes in titles
\let\rmarkdownfootnote\footnote%
\def\footnote{\protect\rmarkdownfootnote}

%%% Change title format to be more compact
\usepackage{titling}

% Create subtitle command for use in maketitle
\newcommand{\subtitle}[1]{
  \posttitle{
    \begin{center}\large#1\end{center}
    }
}

\setlength{\droptitle}{-2em}
  \title{DATA 605 - Discussion 6}
  \pretitle{\vspace{\droptitle}\centering\huge}
  \posttitle{\par}
  \author{Joshua Sturm}
  \preauthor{\centering\large\emph}
  \postauthor{\par}
  \predate{\centering\large\emph}
  \postdate{\par}
  \date{03/07/2018}


\begin{document}
\maketitle

Section 4.1, Exercise 19, Page 152

\section{Problem 19}\label{problem-19}

In a poker hand, John has a very strong hand and bets 5 dollars. The
probability that Mary has a better hand is .04. If Mary had a better
hand she would raise with probability .9, but with a poorer hand she
would only raise with probability .1. If Mary raises, what is the
probability that she has a better hand than John does?

\section{Solution}\label{solution}

Let \(M\) be the event that Mary has a better hand. Then the event that
Mary doesn't have a better hand is simply the complement, ~~i.e.
\(\hspace{8mm}\) \textasciitilde{}\(P(M) = P(M^c)\).

Let \(R\) be the event that Mary raises.

We have \(P(M) = 0.04\), so \(P(M^c) = 1 - 0.04 = 0.96\).

We are also given \(P(R|M) = 0.9\), and \(P(R|M^c) = 0.1\).

We're looking for \(P(M|R)\).

The formula for conditional probability is:
\(P(A|B) = \frac{P(A \cap B)}{P(B)}\).

Our problem becomes \(P(M|R) = \frac{P(M \cap R)}{P(R)}\).

We're missing \(P(R)\). We can use the alternate form of Bayes' Theorem,
rewriting it as:

\(P(M|R) = \frac{P(R|M)P(M)}{P(R|M)P(M) + P(R|M^c)P(M^c)}\)

This simplifies to:
\(\frac{0.9 * 0.04}{(0.9 * 0.04)+(0.1 * 0.96)} = \frac{0.036}{0.132} = 0.\overline{2727}\).


\end{document}
