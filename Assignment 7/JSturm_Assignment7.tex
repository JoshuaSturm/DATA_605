\documentclass[]{article}
\usepackage{lmodern}
\usepackage{amssymb,amsmath}
\usepackage{ifxetex,ifluatex}
\usepackage{fixltx2e} % provides \textsubscript
\ifnum 0\ifxetex 1\fi\ifluatex 1\fi=0 % if pdftex
  \usepackage[T1]{fontenc}
  \usepackage[utf8]{inputenc}
\else % if luatex or xelatex
  \ifxetex
    \usepackage{mathspec}
  \else
    \usepackage{fontspec}
  \fi
  \defaultfontfeatures{Ligatures=TeX,Scale=MatchLowercase}
\fi
% use upquote if available, for straight quotes in verbatim environments
\IfFileExists{upquote.sty}{\usepackage{upquote}}{}
% use microtype if available
\IfFileExists{microtype.sty}{%
\usepackage{microtype}
\UseMicrotypeSet[protrusion]{basicmath} % disable protrusion for tt fonts
}{}
\usepackage[margin=1in]{geometry}
\usepackage{hyperref}
\hypersetup{unicode=true,
            pdftitle={DATA 605 - Assignment 7},
            pdfauthor={Joshua Sturm},
            pdfborder={0 0 0},
            breaklinks=true}
\urlstyle{same}  % don't use monospace font for urls
\usepackage{color}
\usepackage{fancyvrb}
\newcommand{\VerbBar}{|}
\newcommand{\VERB}{\Verb[commandchars=\\\{\}]}
\DefineVerbatimEnvironment{Highlighting}{Verbatim}{commandchars=\\\{\}}
% Add ',fontsize=\small' for more characters per line
\usepackage{framed}
\definecolor{shadecolor}{RGB}{248,248,248}
\newenvironment{Shaded}{\begin{snugshade}}{\end{snugshade}}
\newcommand{\KeywordTok}[1]{\textcolor[rgb]{0.13,0.29,0.53}{\textbf{#1}}}
\newcommand{\DataTypeTok}[1]{\textcolor[rgb]{0.13,0.29,0.53}{#1}}
\newcommand{\DecValTok}[1]{\textcolor[rgb]{0.00,0.00,0.81}{#1}}
\newcommand{\BaseNTok}[1]{\textcolor[rgb]{0.00,0.00,0.81}{#1}}
\newcommand{\FloatTok}[1]{\textcolor[rgb]{0.00,0.00,0.81}{#1}}
\newcommand{\ConstantTok}[1]{\textcolor[rgb]{0.00,0.00,0.00}{#1}}
\newcommand{\CharTok}[1]{\textcolor[rgb]{0.31,0.60,0.02}{#1}}
\newcommand{\SpecialCharTok}[1]{\textcolor[rgb]{0.00,0.00,0.00}{#1}}
\newcommand{\StringTok}[1]{\textcolor[rgb]{0.31,0.60,0.02}{#1}}
\newcommand{\VerbatimStringTok}[1]{\textcolor[rgb]{0.31,0.60,0.02}{#1}}
\newcommand{\SpecialStringTok}[1]{\textcolor[rgb]{0.31,0.60,0.02}{#1}}
\newcommand{\ImportTok}[1]{#1}
\newcommand{\CommentTok}[1]{\textcolor[rgb]{0.56,0.35,0.01}{\textit{#1}}}
\newcommand{\DocumentationTok}[1]{\textcolor[rgb]{0.56,0.35,0.01}{\textbf{\textit{#1}}}}
\newcommand{\AnnotationTok}[1]{\textcolor[rgb]{0.56,0.35,0.01}{\textbf{\textit{#1}}}}
\newcommand{\CommentVarTok}[1]{\textcolor[rgb]{0.56,0.35,0.01}{\textbf{\textit{#1}}}}
\newcommand{\OtherTok}[1]{\textcolor[rgb]{0.56,0.35,0.01}{#1}}
\newcommand{\FunctionTok}[1]{\textcolor[rgb]{0.00,0.00,0.00}{#1}}
\newcommand{\VariableTok}[1]{\textcolor[rgb]{0.00,0.00,0.00}{#1}}
\newcommand{\ControlFlowTok}[1]{\textcolor[rgb]{0.13,0.29,0.53}{\textbf{#1}}}
\newcommand{\OperatorTok}[1]{\textcolor[rgb]{0.81,0.36,0.00}{\textbf{#1}}}
\newcommand{\BuiltInTok}[1]{#1}
\newcommand{\ExtensionTok}[1]{#1}
\newcommand{\PreprocessorTok}[1]{\textcolor[rgb]{0.56,0.35,0.01}{\textit{#1}}}
\newcommand{\AttributeTok}[1]{\textcolor[rgb]{0.77,0.63,0.00}{#1}}
\newcommand{\RegionMarkerTok}[1]{#1}
\newcommand{\InformationTok}[1]{\textcolor[rgb]{0.56,0.35,0.01}{\textbf{\textit{#1}}}}
\newcommand{\WarningTok}[1]{\textcolor[rgb]{0.56,0.35,0.01}{\textbf{\textit{#1}}}}
\newcommand{\AlertTok}[1]{\textcolor[rgb]{0.94,0.16,0.16}{#1}}
\newcommand{\ErrorTok}[1]{\textcolor[rgb]{0.64,0.00,0.00}{\textbf{#1}}}
\newcommand{\NormalTok}[1]{#1}
\usepackage{graphicx,grffile}
\makeatletter
\def\maxwidth{\ifdim\Gin@nat@width>\linewidth\linewidth\else\Gin@nat@width\fi}
\def\maxheight{\ifdim\Gin@nat@height>\textheight\textheight\else\Gin@nat@height\fi}
\makeatother
% Scale images if necessary, so that they will not overflow the page
% margins by default, and it is still possible to overwrite the defaults
% using explicit options in \includegraphics[width, height, ...]{}
\setkeys{Gin}{width=\maxwidth,height=\maxheight,keepaspectratio}
\IfFileExists{parskip.sty}{%
\usepackage{parskip}
}{% else
\setlength{\parindent}{0pt}
\setlength{\parskip}{6pt plus 2pt minus 1pt}
}
\setlength{\emergencystretch}{3em}  % prevent overfull lines
\providecommand{\tightlist}{%
  \setlength{\itemsep}{0pt}\setlength{\parskip}{0pt}}
\setcounter{secnumdepth}{0}
% Redefines (sub)paragraphs to behave more like sections
\ifx\paragraph\undefined\else
\let\oldparagraph\paragraph
\renewcommand{\paragraph}[1]{\oldparagraph{#1}\mbox{}}
\fi
\ifx\subparagraph\undefined\else
\let\oldsubparagraph\subparagraph
\renewcommand{\subparagraph}[1]{\oldsubparagraph{#1}\mbox{}}
\fi

%%% Use protect on footnotes to avoid problems with footnotes in titles
\let\rmarkdownfootnote\footnote%
\def\footnote{\protect\rmarkdownfootnote}

%%% Change title format to be more compact
\usepackage{titling}

% Create subtitle command for use in maketitle
\newcommand{\subtitle}[1]{
  \posttitle{
    \begin{center}\large#1\end{center}
    }
}

\setlength{\droptitle}{-2em}
  \title{DATA 605 - Assignment 7}
  \pretitle{\vspace{\droptitle}\centering\huge}
  \posttitle{\par}
  \author{Joshua Sturm}
  \preauthor{\centering\large\emph}
  \postauthor{\par}
  \predate{\centering\large\emph}
  \postdate{\par}
  \date{March 15, 2018}


\begin{document}
\maketitle

\section{1.}\label{section}

Let \(X_1, X_2, . . . , X_n\) be \(n\) mutually independent random
variables, each of which is uniformly distributed on the integers from
\(1\) to \(k\). Let \(Y\) denote the minimum of the \(X_i\)'s. Find the
distribution of \(Y\).

\subsection{Solution}\label{solution}

We first consider how many ways we can assign \(Y\) to one of the
\(x_i\) variables.

Since there are \(n\) variables, with \(k\) values, there are \(k^n\)
total possibilities.

The number of ways we can get \(Y = 1\) is the total number of
possibilities less the number of possibilities that \(Y\) is
\textit{not} \(1\). So, this can be done in \(k^{n} - (k-1)^n\) ways.

Similarly, the number of ways to get \(Y = 2\) would be
\(k^{n} - (k-2)^{n} - [k^{n} - (k-1)^{n}] = (k-1)^{n} - (k-2)^{n}\).

The pattern we get is the recurrence relation
\(P(Y = m) = \frac{(k-m+1)^{n} - (k-m)^{n}}{k^{n}}\).

\section{2.}\label{section-1}

Your organization owns a copier (future lawyers, etc.) or MRI (future
doctors). This machine has a manufacturer's expected lifetime of 10
years. This means that we expect one failure every ten years. (Include
the probability statements and R Code for each part.)

\subsection{a.}\label{a.}

What is the probability that the machine will fail after 8 years?.
Provide also the expected value and standard deviation. Model as a
geometric. (Hint: the probability is equivalent to not failing during
the first 8 years.)

\subsubsection{Solution}\label{solution-1}

Let \(p\) be the probability that the machine fails, and \(q = 1 - p\)
the probability that it doesn't.

We're looking for the first failure (success) after 8 years, so this is
a geometric distribution. Since we're only expecting one failure in 10
years, \(p = 0.1, \ q = 0.9\).

\(P(X > 8) = 1 - P(X \leq 8)\)

\(1 - (1 - q^{i+1}) = q^{i+1} = 0.9^{9} =\) 0.3874.

\(E[X] = \frac{1}{p} = \frac{1}{0.1} =\) 10.

\(Var(X) = \frac{1-p}{p^2} = \frac{0.9}{(0.1)^2} =\) 90.

Standard deviation = \(\sqrt{Var(X)} = \sqrt{90} \approx\) 9.486833.

\begin{Shaded}
\begin{Highlighting}[]
\NormalTok{pdf <-}\StringTok{ }\KeywordTok{pgeom}\NormalTok{(}\DecValTok{8}\NormalTok{, }\FloatTok{0.1}\NormalTok{, }\DataTypeTok{lower.tail =}\NormalTok{ F)}

\NormalTok{p <-}\StringTok{ }\FloatTok{0.1}
\NormalTok{q <-}\StringTok{ }\DecValTok{1} \OperatorTok{-}\StringTok{ }\NormalTok{p}
\NormalTok{ex <-}\StringTok{ }\NormalTok{p}\OperatorTok{^-}\DecValTok{1}
\NormalTok{var <-}\StringTok{ }\NormalTok{q}\OperatorTok{/}\NormalTok{p}\OperatorTok{^}\DecValTok{2}
\NormalTok{sd <-}\StringTok{ }\KeywordTok{sqrt}\NormalTok{(var)}

\KeywordTok{cat}\NormalTok{(}\KeywordTok{sprintf}\NormalTok{(}\StringTok{"}\CharTok{\textbackslash{}n}\StringTok{ %s = %f }\CharTok{\textbackslash{}n}\StringTok{"}\NormalTok{, }
            \KeywordTok{c}\NormalTok{(}\StringTok{"Probability"}\NormalTok{, }\StringTok{"Expected Value"}\NormalTok{, }\StringTok{"Variance"}\NormalTok{, }\StringTok{"Standard Deviation"}\NormalTok{),}
            \KeywordTok{c}\NormalTok{(pdf, ex, var, sd))}
\NormalTok{    )}
\NormalTok{## }
\NormalTok{##  Probability = 0.387420 }
\NormalTok{##  }
\NormalTok{##  Expected Value = 10.000000 }
\NormalTok{##  }
\NormalTok{##  Variance = 90.000000 }
\NormalTok{##  }
\NormalTok{##  Standard Deviation = 9.486833}
\end{Highlighting}
\end{Shaded}

\subsection{b.}\label{b.}

What is the probability that the machine will fail after 8 years?.
Provide also the expected value and standard deviation. Model as an
exponential.

\subsubsection{Solution}\label{solution-2}

\(P(X > 8) = e^{-\lambda i}\), with \(\lambda = 0.1\).

\(P(X > 8) = e^{-0.8} =\) 0.449329.

\(E[X] = \frac{1}{\lambda} =\) 10.

\(Var(X) = \frac{1}{\lambda^2} = \frac{1}{0.1^2} =\) 100.

Standard Deviation \(= \sqrt{Var(X)} = \sqrt{100} = 10\).

\begin{Shaded}
\begin{Highlighting}[]
\NormalTok{pdf <-}\StringTok{ }\KeywordTok{pexp}\NormalTok{(}\DecValTok{8}\NormalTok{, }\FloatTok{0.1}\NormalTok{, }\DataTypeTok{lower.tail =}\NormalTok{ F)}

\NormalTok{l <-}\StringTok{ }\FloatTok{0.1}
\NormalTok{ex <-}\StringTok{ }\DecValTok{1}\OperatorTok{/}\NormalTok{l}
\NormalTok{var <-}\StringTok{ }\DecValTok{1}\OperatorTok{/}\NormalTok{l}\OperatorTok{^}\DecValTok{2}
\NormalTok{sd <-}\StringTok{ }\KeywordTok{sqrt}\NormalTok{(var)}

\KeywordTok{cat}\NormalTok{(}\KeywordTok{sprintf}\NormalTok{(}\StringTok{"}\CharTok{\textbackslash{}n}\StringTok{ %s = %f }\CharTok{\textbackslash{}n}\StringTok{"}\NormalTok{, }
            \KeywordTok{c}\NormalTok{(}\StringTok{"Probability"}\NormalTok{, }\StringTok{"Expected Value"}\NormalTok{, }\StringTok{"Variance"}\NormalTok{, }\StringTok{"Standard Deviation"}\NormalTok{),}
            \KeywordTok{c}\NormalTok{(pdf, ex, var, sd))}
\NormalTok{    )}
\NormalTok{## }
\NormalTok{##  Probability = 0.449329 }
\NormalTok{##  }
\NormalTok{##  Expected Value = 10.000000 }
\NormalTok{##  }
\NormalTok{##  Variance = 100.000000 }
\NormalTok{##  }
\NormalTok{##  Standard Deviation = 10.000000}
\end{Highlighting}
\end{Shaded}

\subsection{c.}\label{c.}

What is the probability that the machine will fail after 8 years?.
Provide also the expected value and standard deviation. Model as a
binomial. (Hint: 0 success in 8 years)

\subsubsection{Solution}\label{solution-3}

We're looking for 0 successes in 8 years. So, \(P(0)\) with \(n = 8\).

\(P(0) = \binom{8}{0}(0.1)^0 \times 0.9{8-0} = (0.9)^8 =\) 0.4304672.

\(E[X] = np = 8\cdot 0.1 = 0.8\).

\(Var(X) = npq = 0.8\cdot 0.9 =\) 0.72.

Standard Deviation \(= \sqrt{Var(X)} = \sqrt{0.72} =\) 0.8485281.

\begin{Shaded}
\begin{Highlighting}[]
\NormalTok{pdf <-}\StringTok{ }\KeywordTok{pbinom}\NormalTok{(}\DecValTok{0}\NormalTok{, }\DecValTok{8}\NormalTok{, }\FloatTok{0.1}\NormalTok{)}

\NormalTok{n <-}\StringTok{ }\DecValTok{8}
\NormalTok{i <-}\StringTok{ }\DecValTok{0}
\NormalTok{p <-}\StringTok{ }\FloatTok{0.1}
\NormalTok{q <-}\StringTok{ }\FloatTok{0.9}
\NormalTok{ex <-}\StringTok{ }\NormalTok{n}\OperatorTok{*}\NormalTok{p}
\NormalTok{var <-}\StringTok{ }\NormalTok{n}\OperatorTok{*}\NormalTok{p}\OperatorTok{*}\NormalTok{q}
\NormalTok{sd <-}\StringTok{ }\KeywordTok{sqrt}\NormalTok{(var)}

\KeywordTok{cat}\NormalTok{(}\KeywordTok{sprintf}\NormalTok{(}\StringTok{"}\CharTok{\textbackslash{}n}\StringTok{ %s = %f }\CharTok{\textbackslash{}n}\StringTok{"}\NormalTok{, }
            \KeywordTok{c}\NormalTok{(}\StringTok{"Probability"}\NormalTok{, }\StringTok{"Expected Value"}\NormalTok{, }\StringTok{"Variance"}\NormalTok{, }\StringTok{"Standard Deviation"}\NormalTok{),}
            \KeywordTok{c}\NormalTok{(pdf, ex, var, sd))}
\NormalTok{    )}
\NormalTok{## }
\NormalTok{##  Probability = 0.430467 }
\NormalTok{##  }
\NormalTok{##  Expected Value = 0.800000 }
\NormalTok{##  }
\NormalTok{##  Variance = 0.720000 }
\NormalTok{##  }
\NormalTok{##  Standard Deviation = 0.848528}
\end{Highlighting}
\end{Shaded}

\subsection{d.}\label{d.}

What is the probability that the machine will fail after 8 years?.
Provide also the expected value and standard deviation. Model as a
Poisson.

\subsubsection{Solution}\label{solution-4}

Since Poisson uses averages, and we expect one failure every 10 years,
we can say the average yearly failure rate is 0.1. We're looking for 0
failures in the first 8 years.

\(P_{i}(t) = \frac{(\lambda t)^{i}e^{-\lambda t}}{i!}\)

\(P_{0}(8) = \frac{(0.1\cdot 8)^{0}e^{-0.1\cdot 8}}{0!} = e^{-0.8} =\)
0.449329.

\(E[X] = Var(X) = \lambda\cdot t = 0.1 \cdot 8 = 0.8\).

Standard Deviation \(= \sqrt{Var(X)} = \sqrt{0.8} =\) 0.8944272.

\begin{Shaded}
\begin{Highlighting}[]
\NormalTok{lambda <-}\StringTok{ }\FloatTok{0.1}
\NormalTok{t <-}\StringTok{ }\DecValTok{8}
\NormalTok{i <-}\StringTok{ }\DecValTok{0}
\NormalTok{ex <-}\StringTok{ }\NormalTok{lambda}\OperatorTok{*}\NormalTok{t}
\NormalTok{var <-}\StringTok{ }\NormalTok{lambda}\OperatorTok{*}\NormalTok{t}
\NormalTok{sd <-}\StringTok{ }\KeywordTok{sqrt}\NormalTok{(var)}
\NormalTok{pdf <-}\StringTok{ }\KeywordTok{ppois}\NormalTok{(i, t}\OperatorTok{*}\NormalTok{lambda)}

\KeywordTok{cat}\NormalTok{(}\KeywordTok{sprintf}\NormalTok{(}\StringTok{"}\CharTok{\textbackslash{}n}\StringTok{ %s = %f }\CharTok{\textbackslash{}n}\StringTok{"}\NormalTok{, }
            \KeywordTok{c}\NormalTok{(}\StringTok{"Probability"}\NormalTok{, }\StringTok{"Expected Value"}\NormalTok{, }\StringTok{"Variance"}\NormalTok{, }\StringTok{"Standard Deviation"}\NormalTok{),}
            \KeywordTok{c}\NormalTok{(pdf, ex, var, sd))}
\NormalTok{    )}
\NormalTok{## }
\NormalTok{##  Probability = 0.449329 }
\NormalTok{##  }
\NormalTok{##  Expected Value = 0.800000 }
\NormalTok{##  }
\NormalTok{##  Variance = 0.800000 }
\NormalTok{##  }
\NormalTok{##  Standard Deviation = 0.894427}
\end{Highlighting}
\end{Shaded}


\end{document}
