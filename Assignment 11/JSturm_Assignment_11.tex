\documentclass[]{article}
\usepackage{lmodern}
\usepackage{amssymb,amsmath}
\usepackage{ifxetex,ifluatex}
\usepackage{fixltx2e} % provides \textsubscript
\ifnum 0\ifxetex 1\fi\ifluatex 1\fi=0 % if pdftex
  \usepackage[T1]{fontenc}
  \usepackage[utf8]{inputenc}
\else % if luatex or xelatex
  \ifxetex
    \usepackage{mathspec}
  \else
    \usepackage{fontspec}
  \fi
  \defaultfontfeatures{Ligatures=TeX,Scale=MatchLowercase}
\fi
% use upquote if available, for straight quotes in verbatim environments
\IfFileExists{upquote.sty}{\usepackage{upquote}}{}
% use microtype if available
\IfFileExists{microtype.sty}{%
\usepackage{microtype}
\UseMicrotypeSet[protrusion]{basicmath} % disable protrusion for tt fonts
}{}
\usepackage[margin=1in]{geometry}
\usepackage{hyperref}
\hypersetup{unicode=true,
            pdftitle={DATA 605 - Assignment 11},
            pdfauthor={Joshua Sturm},
            pdfborder={0 0 0},
            breaklinks=true}
\urlstyle{same}  % don't use monospace font for urls
\usepackage{color}
\usepackage{fancyvrb}
\newcommand{\VerbBar}{|}
\newcommand{\VERB}{\Verb[commandchars=\\\{\}]}
\DefineVerbatimEnvironment{Highlighting}{Verbatim}{commandchars=\\\{\}}
% Add ',fontsize=\small' for more characters per line
\usepackage{framed}
\definecolor{shadecolor}{RGB}{248,248,248}
\newenvironment{Shaded}{\begin{snugshade}}{\end{snugshade}}
\newcommand{\KeywordTok}[1]{\textcolor[rgb]{0.13,0.29,0.53}{\textbf{#1}}}
\newcommand{\DataTypeTok}[1]{\textcolor[rgb]{0.13,0.29,0.53}{#1}}
\newcommand{\DecValTok}[1]{\textcolor[rgb]{0.00,0.00,0.81}{#1}}
\newcommand{\BaseNTok}[1]{\textcolor[rgb]{0.00,0.00,0.81}{#1}}
\newcommand{\FloatTok}[1]{\textcolor[rgb]{0.00,0.00,0.81}{#1}}
\newcommand{\ConstantTok}[1]{\textcolor[rgb]{0.00,0.00,0.00}{#1}}
\newcommand{\CharTok}[1]{\textcolor[rgb]{0.31,0.60,0.02}{#1}}
\newcommand{\SpecialCharTok}[1]{\textcolor[rgb]{0.00,0.00,0.00}{#1}}
\newcommand{\StringTok}[1]{\textcolor[rgb]{0.31,0.60,0.02}{#1}}
\newcommand{\VerbatimStringTok}[1]{\textcolor[rgb]{0.31,0.60,0.02}{#1}}
\newcommand{\SpecialStringTok}[1]{\textcolor[rgb]{0.31,0.60,0.02}{#1}}
\newcommand{\ImportTok}[1]{#1}
\newcommand{\CommentTok}[1]{\textcolor[rgb]{0.56,0.35,0.01}{\textit{#1}}}
\newcommand{\DocumentationTok}[1]{\textcolor[rgb]{0.56,0.35,0.01}{\textbf{\textit{#1}}}}
\newcommand{\AnnotationTok}[1]{\textcolor[rgb]{0.56,0.35,0.01}{\textbf{\textit{#1}}}}
\newcommand{\CommentVarTok}[1]{\textcolor[rgb]{0.56,0.35,0.01}{\textbf{\textit{#1}}}}
\newcommand{\OtherTok}[1]{\textcolor[rgb]{0.56,0.35,0.01}{#1}}
\newcommand{\FunctionTok}[1]{\textcolor[rgb]{0.00,0.00,0.00}{#1}}
\newcommand{\VariableTok}[1]{\textcolor[rgb]{0.00,0.00,0.00}{#1}}
\newcommand{\ControlFlowTok}[1]{\textcolor[rgb]{0.13,0.29,0.53}{\textbf{#1}}}
\newcommand{\OperatorTok}[1]{\textcolor[rgb]{0.81,0.36,0.00}{\textbf{#1}}}
\newcommand{\BuiltInTok}[1]{#1}
\newcommand{\ExtensionTok}[1]{#1}
\newcommand{\PreprocessorTok}[1]{\textcolor[rgb]{0.56,0.35,0.01}{\textit{#1}}}
\newcommand{\AttributeTok}[1]{\textcolor[rgb]{0.77,0.63,0.00}{#1}}
\newcommand{\RegionMarkerTok}[1]{#1}
\newcommand{\InformationTok}[1]{\textcolor[rgb]{0.56,0.35,0.01}{\textbf{\textit{#1}}}}
\newcommand{\WarningTok}[1]{\textcolor[rgb]{0.56,0.35,0.01}{\textbf{\textit{#1}}}}
\newcommand{\AlertTok}[1]{\textcolor[rgb]{0.94,0.16,0.16}{#1}}
\newcommand{\ErrorTok}[1]{\textcolor[rgb]{0.64,0.00,0.00}{\textbf{#1}}}
\newcommand{\NormalTok}[1]{#1}
\usepackage{graphicx,grffile}
\makeatletter
\def\maxwidth{\ifdim\Gin@nat@width>\linewidth\linewidth\else\Gin@nat@width\fi}
\def\maxheight{\ifdim\Gin@nat@height>\textheight\textheight\else\Gin@nat@height\fi}
\makeatother
% Scale images if necessary, so that they will not overflow the page
% margins by default, and it is still possible to overwrite the defaults
% using explicit options in \includegraphics[width, height, ...]{}
\setkeys{Gin}{width=\maxwidth,height=\maxheight,keepaspectratio}
\IfFileExists{parskip.sty}{%
\usepackage{parskip}
}{% else
\setlength{\parindent}{0pt}
\setlength{\parskip}{6pt plus 2pt minus 1pt}
}
\setlength{\emergencystretch}{3em}  % prevent overfull lines
\providecommand{\tightlist}{%
  \setlength{\itemsep}{0pt}\setlength{\parskip}{0pt}}
\setcounter{secnumdepth}{0}
% Redefines (sub)paragraphs to behave more like sections
\ifx\paragraph\undefined\else
\let\oldparagraph\paragraph
\renewcommand{\paragraph}[1]{\oldparagraph{#1}\mbox{}}
\fi
\ifx\subparagraph\undefined\else
\let\oldsubparagraph\subparagraph
\renewcommand{\subparagraph}[1]{\oldsubparagraph{#1}\mbox{}}
\fi

%%% Use protect on footnotes to avoid problems with footnotes in titles
\let\rmarkdownfootnote\footnote%
\def\footnote{\protect\rmarkdownfootnote}

%%% Change title format to be more compact
\usepackage{titling}

% Create subtitle command for use in maketitle
\newcommand{\subtitle}[1]{
  \posttitle{
    \begin{center}\large#1\end{center}
    }
}

\setlength{\droptitle}{-2em}
  \title{DATA 605 - Assignment 11}
  \pretitle{\vspace{\droptitle}\centering\huge}
  \posttitle{\par}
  \author{Joshua Sturm}
  \preauthor{\centering\large\emph}
  \postauthor{\par}
  \predate{\centering\large\emph}
  \postdate{\par}
  \date{April 22, 2018}


\begin{document}
\maketitle

Using the \texttt{cars} dataset in R, build a linear model for stopping
distance as a function of speed and replicate the analysis of your
textbook chapter 3 (visualization, quality evaluation of the model, and
residual analysis.)

\begin{Shaded}
\begin{Highlighting}[]
\KeywordTok{library}\NormalTok{(tidyverse)}
\KeywordTok{library}\NormalTok{(gridExtra)}
\end{Highlighting}
\end{Shaded}

\begin{Shaded}
\begin{Highlighting}[]
\NormalTok{data <-}\StringTok{ }\NormalTok{cars}
\KeywordTok{glimpse}\NormalTok{(data)}
\NormalTok{## Observations: 50}
\NormalTok{## Variables: 2}
\NormalTok{## $ speed <dbl> 4, 4, 7, 7, 8, 9, 10, 10, 10, 11, 11, 12, 12, 12, 12, 13...}
\NormalTok{## $ dist  <dbl> 2, 10, 4, 22, 16, 10, 18, 26, 34, 17, 28, 14, 20, 24, 28...}
\end{Highlighting}
\end{Shaded}

The dataset contains 2 variables, and 50 cases.

\begin{Shaded}
\begin{Highlighting}[]
\NormalTok{model <-}\StringTok{ }\KeywordTok{lm}\NormalTok{(}\DataTypeTok{formula =}\NormalTok{ dist }\OperatorTok{~}\StringTok{ }\NormalTok{speed,}
            \DataTypeTok{data =}\NormalTok{ data)}
\KeywordTok{summary}\NormalTok{(model)}
\NormalTok{## }
\NormalTok{## Call:}
\NormalTok{## lm(formula = dist ~ speed, data = data)}
\NormalTok{## }
\NormalTok{## Residuals:}
\NormalTok{##     Min      1Q  Median      3Q     Max }
\NormalTok{## -29.069  -9.525  -2.272   9.215  43.201 }
\NormalTok{## }
\NormalTok{## Coefficients:}
\NormalTok{##             Estimate Std. Error t value Pr(>|t|)    }
\NormalTok{## (Intercept) -17.5791     6.7584  -2.601   0.0123 *  }
\NormalTok{## speed         3.9324     0.4155   9.464 1.49e-12 ***}
\NormalTok{## ---}
\NormalTok{## Signif. codes:  0 '***' 0.001 '**' 0.01 '*' 0.05 '.' 0.1 ' ' 1}
\NormalTok{## }
\NormalTok{## Residual standard error: 15.38 on 48 degrees of freedom}
\NormalTok{## Multiple R-squared:  0.6511, Adjusted R-squared:  0.6438 }
\NormalTok{## F-statistic: 89.57 on 1 and 48 DF,  p-value: 1.49e-12}
\end{Highlighting}
\end{Shaded}

The model has an adjusted r-squared of 0.6438102, and a p-value of
\(\approx\) 0.

\begin{Shaded}
\begin{Highlighting}[]

\NormalTok{rp1 <-}\StringTok{ }\KeywordTok{ggplot}\NormalTok{(model, }\KeywordTok{aes}\NormalTok{(.fitted, .resid)) }\OperatorTok{+}
\StringTok{  }\KeywordTok{geom_point}\NormalTok{() }\OperatorTok{+}
\StringTok{  }\KeywordTok{geom_hline}\NormalTok{(}\DataTypeTok{yintercept =} \DecValTok{0}\NormalTok{) }\OperatorTok{+}
\StringTok{  }\KeywordTok{geom_smooth}\NormalTok{(}\DataTypeTok{se =} \OtherTok{FALSE}\NormalTok{) }\OperatorTok{+}
\StringTok{  }\KeywordTok{labs}\NormalTok{(}\DataTypeTok{title =} \StringTok{"Residuals vs Fitted"}\NormalTok{)}

\NormalTok{rp2 <-}\StringTok{ }\KeywordTok{ggplot}\NormalTok{(model, }\KeywordTok{aes}\NormalTok{(.fitted, .stdresid)) }\OperatorTok{+}
\StringTok{  }\KeywordTok{geom_point}\NormalTok{() }\OperatorTok{+}
\StringTok{  }\KeywordTok{geom_hline}\NormalTok{(}\DataTypeTok{yintercept =} \DecValTok{0}\NormalTok{) }\OperatorTok{+}
\StringTok{  }\KeywordTok{geom_smooth}\NormalTok{(}\DataTypeTok{se =} \OtherTok{FALSE}\NormalTok{)}

\NormalTok{rp3 <-}\StringTok{ }\KeywordTok{ggplot}\NormalTok{(model) }\OperatorTok{+}
\StringTok{  }\KeywordTok{stat_qq}\NormalTok{(}\KeywordTok{aes}\NormalTok{(}\DataTypeTok{sample =}\NormalTok{ .stdresid)) }\OperatorTok{+}
\StringTok{  }\KeywordTok{geom_abline}\NormalTok{()}

\NormalTok{rp4 <-}\StringTok{ }\KeywordTok{ggplot}\NormalTok{(model, }\KeywordTok{aes}\NormalTok{(.fitted, }\KeywordTok{sqrt}\NormalTok{(}\KeywordTok{abs}\NormalTok{(.stdresid)))) }\OperatorTok{+}
\StringTok{  }\KeywordTok{geom_point}\NormalTok{() }\OperatorTok{+}
\StringTok{  }\KeywordTok{geom_smooth}\NormalTok{(}\DataTypeTok{se =} \OtherTok{FALSE}\NormalTok{) }\OperatorTok{+}
\StringTok{  }\KeywordTok{labs}\NormalTok{(}\DataTypeTok{title =} \StringTok{"Scale-Location"}\NormalTok{)}

\NormalTok{rp5 <-}\StringTok{ }\KeywordTok{ggplot}\NormalTok{(model, }\KeywordTok{aes}\NormalTok{(}\KeywordTok{seq_along}\NormalTok{(.cooksd), .cooksd)) }\OperatorTok{+}
\StringTok{  }\KeywordTok{geom_col}\NormalTok{() }\OperatorTok{+}
\StringTok{  }\KeywordTok{ylim}\NormalTok{(}\DecValTok{0}\NormalTok{, }\FloatTok{0.0075}\NormalTok{) }\OperatorTok{+}
\StringTok{  }\KeywordTok{labs}\NormalTok{(}\DataTypeTok{title =} \StringTok{"Cook's Distance"}\NormalTok{)}

\NormalTok{rp6 <-}\StringTok{ }\KeywordTok{ggplot}\NormalTok{(model, }\KeywordTok{aes}\NormalTok{(.hat, .stdresid)) }\OperatorTok{+}
\StringTok{  }\KeywordTok{geom_point}\NormalTok{(}\KeywordTok{aes}\NormalTok{(}\DataTypeTok{size =}\NormalTok{ .cooksd)) }\OperatorTok{+}
\StringTok{  }\KeywordTok{geom_smooth}\NormalTok{(}\DataTypeTok{se =} \OtherTok{FALSE}\NormalTok{, }\DataTypeTok{size =} \FloatTok{0.5}\NormalTok{) }\OperatorTok{+}
\StringTok{  }\KeywordTok{labs}\NormalTok{(}\DataTypeTok{title =} \StringTok{"Residuals vs Leverage"}\NormalTok{)}

\NormalTok{rp7 <-}\StringTok{ }\KeywordTok{ggplot}\NormalTok{(model, }\KeywordTok{aes}\NormalTok{(.hat, .cooksd)) }\OperatorTok{+}
\StringTok{  }\KeywordTok{geom_vline}\NormalTok{(}\DataTypeTok{xintercept =} \DecValTok{0}\NormalTok{, }\DataTypeTok{colour =} \OtherTok{NA}\NormalTok{) }\OperatorTok{+}
\StringTok{  }\KeywordTok{geom_abline}\NormalTok{(}\DataTypeTok{slope =} \KeywordTok{seq}\NormalTok{(}\DecValTok{0}\NormalTok{, }\DecValTok{3}\NormalTok{, }\DataTypeTok{by =} \FloatTok{0.5}\NormalTok{), }\DataTypeTok{colour =} \StringTok{"white"}\NormalTok{) }\OperatorTok{+}
\StringTok{  }\KeywordTok{geom_smooth}\NormalTok{(}\DataTypeTok{se =} \OtherTok{FALSE}\NormalTok{) }\OperatorTok{+}
\StringTok{  }\KeywordTok{geom_point}\NormalTok{() }\OperatorTok{+}
\StringTok{  }\KeywordTok{labs}\NormalTok{(}\DataTypeTok{title =} \StringTok{"Cook's distance vs Leverage"}\NormalTok{)}

\NormalTok{rp8 <-}\StringTok{ }\KeywordTok{ggplot}\NormalTok{(model, }\KeywordTok{aes}\NormalTok{(.resid)) }\OperatorTok{+}
\StringTok{  }\KeywordTok{geom_histogram}\NormalTok{(}\DataTypeTok{bins =} \DecValTok{7}\NormalTok{, }\DataTypeTok{color=}\StringTok{"darkblue"}\NormalTok{, }\DataTypeTok{fill=}\StringTok{"steelblue"}\NormalTok{)}

\KeywordTok{grid.arrange}\NormalTok{(rp1, rp2, rp3, rp4, rp5, rp6, rp7, rp8, }\DataTypeTok{ncol =} \DecValTok{2}\NormalTok{)}
\end{Highlighting}
\end{Shaded}

\includegraphics{JSturm_Assignment_11_files/figure-latex/predict-plots-1.pdf}

The residuals appear to be nearly normal, as can be seen in the plots
\texttt{fitted\ vs.\ residuals}, \texttt{Q-Q\ Plot}, and the residual
histogram. Taken together with the information from the model summary,
we can conclude that this model is sufficiently capable of making
predictions on this dataset.


\end{document}
