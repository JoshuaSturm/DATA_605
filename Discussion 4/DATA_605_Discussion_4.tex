\documentclass[]{article}
\usepackage{lmodern}
\usepackage{amssymb,amsmath}
\usepackage{ifxetex,ifluatex}
\usepackage{fixltx2e} % provides \textsubscript
\ifnum 0\ifxetex 1\fi\ifluatex 1\fi=0 % if pdftex
  \usepackage[T1]{fontenc}
  \usepackage[utf8]{inputenc}
\else % if luatex or xelatex
  \ifxetex
    \usepackage{mathspec}
  \else
    \usepackage{fontspec}
  \fi
  \defaultfontfeatures{Ligatures=TeX,Scale=MatchLowercase}
\fi
% use upquote if available, for straight quotes in verbatim environments
\IfFileExists{upquote.sty}{\usepackage{upquote}}{}
% use microtype if available
\IfFileExists{microtype.sty}{%
\usepackage{microtype}
\UseMicrotypeSet[protrusion]{basicmath} % disable protrusion for tt fonts
}{}
\usepackage[margin=1in]{geometry}
\usepackage{hyperref}
\hypersetup{unicode=true,
            pdftitle={DATA 605 - Discussion 4},
            pdfauthor={Joshua Sturm},
            pdfborder={0 0 0},
            breaklinks=true}
\urlstyle{same}  % don't use monospace font for urls
\usepackage{graphicx,grffile}
\makeatletter
\def\maxwidth{\ifdim\Gin@nat@width>\linewidth\linewidth\else\Gin@nat@width\fi}
\def\maxheight{\ifdim\Gin@nat@height>\textheight\textheight\else\Gin@nat@height\fi}
\makeatother
% Scale images if necessary, so that they will not overflow the page
% margins by default, and it is still possible to overwrite the defaults
% using explicit options in \includegraphics[width, height, ...]{}
\setkeys{Gin}{width=\maxwidth,height=\maxheight,keepaspectratio}
\IfFileExists{parskip.sty}{%
\usepackage{parskip}
}{% else
\setlength{\parindent}{0pt}
\setlength{\parskip}{6pt plus 2pt minus 1pt}
}
\setlength{\emergencystretch}{3em}  % prevent overfull lines
\providecommand{\tightlist}{%
  \setlength{\itemsep}{0pt}\setlength{\parskip}{0pt}}
\setcounter{secnumdepth}{0}
% Redefines (sub)paragraphs to behave more like sections
\ifx\paragraph\undefined\else
\let\oldparagraph\paragraph
\renewcommand{\paragraph}[1]{\oldparagraph{#1}\mbox{}}
\fi
\ifx\subparagraph\undefined\else
\let\oldsubparagraph\subparagraph
\renewcommand{\subparagraph}[1]{\oldsubparagraph{#1}\mbox{}}
\fi

%%% Use protect on footnotes to avoid problems with footnotes in titles
\let\rmarkdownfootnote\footnote%
\def\footnote{\protect\rmarkdownfootnote}

%%% Change title format to be more compact
\usepackage{titling}

% Create subtitle command for use in maketitle
\newcommand{\subtitle}[1]{
  \posttitle{
    \begin{center}\large#1\end{center}
    }
}

\setlength{\droptitle}{-2em}
  \title{DATA 605 - Discussion 4}
  \pretitle{\vspace{\droptitle}\centering\huge}
  \posttitle{\par}
  \author{Joshua Sturm}
  \preauthor{\centering\large\emph}
  \postauthor{\par}
  \predate{\centering\large\emph}
  \postdate{\par}
  \date{02/20/2018}


\begin{document}
\maketitle

Page 443, Exercise C26

\subsection{Verify that the function below is a linear
transformation.}\label{verify-that-the-function-below-is-a-linear-transformation.}

\(T = P_2 \to \mathbb{C}^2, \hspace{5mm} T(a + bx + cx^2) = \begin{bmatrix}2a - b\\ b + c\end{bmatrix}\)

The function must have the following two properties to be a linear
transformation:

\(1. \hspace{2mm} T(\textbf{u}_1 + \textbf{u}_2) = T(\textbf{u}_1) + T(\textbf{u}_2) \hspace{5mm} \forall \hspace{2mm} \textbf{u}_1, \textbf{u}_2 \in U\)

\(2. \hspace{2mm} T(\alpha \textbf{u}) = \alpha T(\textbf{u}) \hspace{5mm} \forall \hspace{2mm} \textbf{u} \in U \text{ and } \forall \hspace{2mm} \alpha \in \mathbb{C}\)

\subsubsection{Condition 1}\label{condition-1}

We can picky dummy variables for \(\textbf{v}\), e.g.~let
\(\textbf{v} = d + ex + fx^2\).

\(T\Big[(a + bx + cx^2) + (d + ex + fx^2)\Big] = T\Big[(a + d) + (bx + ex) + (cx^2 + fx^2)\Big]\).

Factor out the
\(x: \hspace{2mm} T\Big[(a + d) + x(b + e) + x^2(c + f)\Big]\).

We can now check if our transformation will equal
\(\begin{bmatrix}2a - b\\ b + c\end{bmatrix}\).

\(\begin{bmatrix}2(a + d) - (b + e)\\ (b + e) + (c + f)\end{bmatrix} = \begin{bmatrix}2a + 2d - b + e \\ (b + e) + (c + f)\end{bmatrix}\).

\(\begin{bmatrix}(2a - b) + (2d - e)\\ (b + e) + (c + f)\end{bmatrix}\).

Since it's associative:
\(\begin{bmatrix}2a - b \\ b + c\end{bmatrix} + \begin{bmatrix}2d - e\\ e + f\end{bmatrix}\).

\(= T(\textbf{u}) + T(\textbf{v})\).

\subsubsection{Condition 2}\label{condition-2}

\(T\Big[\alpha(a + bx + cx^2)\Big] = T\Big[(\alpha a) + (\alpha bx) + (\alpha cx^2)\Big]\).

Factor out the
\(x: T\Big[(\alpha a) + x(\alpha b) + x^2(\alpha cx)\Big]\).

Following the same procedure as in the first part:
\(\begin{bmatrix}2(\alpha a) - (\alpha b)\\ (\alpha b) + (\alpha c)\end{bmatrix}\).

Factor our
\(\alpha: \begin{bmatrix}\alpha(2a - b)\\ \alpha(b + c)\end{bmatrix} = \alpha \begin{bmatrix}2a - b\\ b + c\end{bmatrix}\).

\(= \alpha T(\textbf{u})\).

Since both conditions are satisfied, we can conclude that this function
is a linear transformation.


\end{document}
