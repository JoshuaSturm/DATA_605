\documentclass[]{article}
\usepackage{lmodern}
\usepackage{amssymb,amsmath}
\usepackage{ifxetex,ifluatex}
\usepackage{fixltx2e} % provides \textsubscript
\ifnum 0\ifxetex 1\fi\ifluatex 1\fi=0 % if pdftex
  \usepackage[T1]{fontenc}
  \usepackage[utf8]{inputenc}
\else % if luatex or xelatex
  \ifxetex
    \usepackage{mathspec}
  \else
    \usepackage{fontspec}
  \fi
  \defaultfontfeatures{Ligatures=TeX,Scale=MatchLowercase}
\fi
% use upquote if available, for straight quotes in verbatim environments
\IfFileExists{upquote.sty}{\usepackage{upquote}}{}
% use microtype if available
\IfFileExists{microtype.sty}{%
\usepackage{microtype}
\UseMicrotypeSet[protrusion]{basicmath} % disable protrusion for tt fonts
}{}
\usepackage[margin=1in]{geometry}
\usepackage{hyperref}
\hypersetup{unicode=true,
            pdftitle={DATA 605 - Assignment 1},
            pdfauthor={Joshua Sturm},
            pdfborder={0 0 0},
            breaklinks=true}
\urlstyle{same}  % don't use monospace font for urls
\usepackage{color}
\usepackage{fancyvrb}
\newcommand{\VerbBar}{|}
\newcommand{\VERB}{\Verb[commandchars=\\\{\}]}
\DefineVerbatimEnvironment{Highlighting}{Verbatim}{commandchars=\\\{\}}
% Add ',fontsize=\small' for more characters per line
\usepackage{framed}
\definecolor{shadecolor}{RGB}{248,248,248}
\newenvironment{Shaded}{\begin{snugshade}}{\end{snugshade}}
\newcommand{\KeywordTok}[1]{\textcolor[rgb]{0.13,0.29,0.53}{\textbf{#1}}}
\newcommand{\DataTypeTok}[1]{\textcolor[rgb]{0.13,0.29,0.53}{#1}}
\newcommand{\DecValTok}[1]{\textcolor[rgb]{0.00,0.00,0.81}{#1}}
\newcommand{\BaseNTok}[1]{\textcolor[rgb]{0.00,0.00,0.81}{#1}}
\newcommand{\FloatTok}[1]{\textcolor[rgb]{0.00,0.00,0.81}{#1}}
\newcommand{\ConstantTok}[1]{\textcolor[rgb]{0.00,0.00,0.00}{#1}}
\newcommand{\CharTok}[1]{\textcolor[rgb]{0.31,0.60,0.02}{#1}}
\newcommand{\SpecialCharTok}[1]{\textcolor[rgb]{0.00,0.00,0.00}{#1}}
\newcommand{\StringTok}[1]{\textcolor[rgb]{0.31,0.60,0.02}{#1}}
\newcommand{\VerbatimStringTok}[1]{\textcolor[rgb]{0.31,0.60,0.02}{#1}}
\newcommand{\SpecialStringTok}[1]{\textcolor[rgb]{0.31,0.60,0.02}{#1}}
\newcommand{\ImportTok}[1]{#1}
\newcommand{\CommentTok}[1]{\textcolor[rgb]{0.56,0.35,0.01}{\textit{#1}}}
\newcommand{\DocumentationTok}[1]{\textcolor[rgb]{0.56,0.35,0.01}{\textbf{\textit{#1}}}}
\newcommand{\AnnotationTok}[1]{\textcolor[rgb]{0.56,0.35,0.01}{\textbf{\textit{#1}}}}
\newcommand{\CommentVarTok}[1]{\textcolor[rgb]{0.56,0.35,0.01}{\textbf{\textit{#1}}}}
\newcommand{\OtherTok}[1]{\textcolor[rgb]{0.56,0.35,0.01}{#1}}
\newcommand{\FunctionTok}[1]{\textcolor[rgb]{0.00,0.00,0.00}{#1}}
\newcommand{\VariableTok}[1]{\textcolor[rgb]{0.00,0.00,0.00}{#1}}
\newcommand{\ControlFlowTok}[1]{\textcolor[rgb]{0.13,0.29,0.53}{\textbf{#1}}}
\newcommand{\OperatorTok}[1]{\textcolor[rgb]{0.81,0.36,0.00}{\textbf{#1}}}
\newcommand{\BuiltInTok}[1]{#1}
\newcommand{\ExtensionTok}[1]{#1}
\newcommand{\PreprocessorTok}[1]{\textcolor[rgb]{0.56,0.35,0.01}{\textit{#1}}}
\newcommand{\AttributeTok}[1]{\textcolor[rgb]{0.77,0.63,0.00}{#1}}
\newcommand{\RegionMarkerTok}[1]{#1}
\newcommand{\InformationTok}[1]{\textcolor[rgb]{0.56,0.35,0.01}{\textbf{\textit{#1}}}}
\newcommand{\WarningTok}[1]{\textcolor[rgb]{0.56,0.35,0.01}{\textbf{\textit{#1}}}}
\newcommand{\AlertTok}[1]{\textcolor[rgb]{0.94,0.16,0.16}{#1}}
\newcommand{\ErrorTok}[1]{\textcolor[rgb]{0.64,0.00,0.00}{\textbf{#1}}}
\newcommand{\NormalTok}[1]{#1}
\usepackage{graphicx,grffile}
\makeatletter
\def\maxwidth{\ifdim\Gin@nat@width>\linewidth\linewidth\else\Gin@nat@width\fi}
\def\maxheight{\ifdim\Gin@nat@height>\textheight\textheight\else\Gin@nat@height\fi}
\makeatother
% Scale images if necessary, so that they will not overflow the page
% margins by default, and it is still possible to overwrite the defaults
% using explicit options in \includegraphics[width, height, ...]{}
\setkeys{Gin}{width=\maxwidth,height=\maxheight,keepaspectratio}
\IfFileExists{parskip.sty}{%
\usepackage{parskip}
}{% else
\setlength{\parindent}{0pt}
\setlength{\parskip}{6pt plus 2pt minus 1pt}
}
\setlength{\emergencystretch}{3em}  % prevent overfull lines
\providecommand{\tightlist}{%
  \setlength{\itemsep}{0pt}\setlength{\parskip}{0pt}}
\setcounter{secnumdepth}{0}
% Redefines (sub)paragraphs to behave more like sections
\ifx\paragraph\undefined\else
\let\oldparagraph\paragraph
\renewcommand{\paragraph}[1]{\oldparagraph{#1}\mbox{}}
\fi
\ifx\subparagraph\undefined\else
\let\oldsubparagraph\subparagraph
\renewcommand{\subparagraph}[1]{\oldsubparagraph{#1}\mbox{}}
\fi

%%% Use protect on footnotes to avoid problems with footnotes in titles
\let\rmarkdownfootnote\footnote%
\def\footnote{\protect\rmarkdownfootnote}

%%% Change title format to be more compact
\usepackage{titling}

% Create subtitle command for use in maketitle
\newcommand{\subtitle}[1]{
  \posttitle{
    \begin{center}\large#1\end{center}
    }
}

\setlength{\droptitle}{-2em}
  \title{DATA 605 - Assignment 1}
  \pretitle{\vspace{\droptitle}\centering\huge}
  \posttitle{\par}
  \author{Joshua Sturm}
  \preauthor{\centering\large\emph}
  \postauthor{\par}
  \predate{\centering\large\emph}
  \postdate{\par}
  \date{02/01/2018}


\begin{document}
\maketitle

\section{Problem Set 1}\label{problem-set-1}

\subsection{1. Calculate the dot product of u,v where u={[}0.5;0.5{]}
and
v={[}3;-4{]}}\label{calculate-the-dot-product-of-uv-where-u0.50.5-and-v3-4}

\subsubsection{Solution}\label{solution}

\[ \textbf{u} \cdot \textbf{v} = \textbf{u}_1\cdot \textbf{v}_1 + \textbf{u}_2\cdot \textbf{v}_2 \\
0.5\cdot 3 + 0.5\cdot -4 \to 1.5 - 2 = -0.5\]

\begin{Shaded}
\begin{Highlighting}[]
\NormalTok{u <-}\StringTok{ }\KeywordTok{c}\NormalTok{(}\FloatTok{0.5}\NormalTok{, }\FloatTok{0.5}\NormalTok{)}
\NormalTok{v <-}\StringTok{ }\KeywordTok{c}\NormalTok{(}\DecValTok{3}\NormalTok{,}\OperatorTok{-}\DecValTok{4}\NormalTok{)}
\NormalTok{udotv <-}\StringTok{ }\KeywordTok{drop}\NormalTok{(u }\OperatorTok\StringTok{ }\NormalTok{v) }\CommentTok{# The drop function is used to remove dimensions > 1}
\NormalTok{udotv}
\end{Highlighting}
\end{Shaded}

\begin{verbatim}
## [1] -0.5
\end{verbatim}

\subsection{2. What are the lengths of u and
v?}\label{what-are-the-lengths-of-u-and-v}

\subsubsection{Solution}\label{solution-1}

\[ ||\textbf{u}|| = \sqrt{\textbf{u}_1^2 + \textbf{u}_2^2 + ... + \textbf{u}_n^2} \\
\sqrt{0.5^2 + 0.5^2} = \sqrt{0.25 + 0.25} = \sqrt{0.5} = 0.7071068\].

\[ ||\textbf{v}|| = \sqrt{3^2 + (-4)^2} = \sqrt{9 + 16} = \sqrt{25} = 5\].

\begin{Shaded}
\begin{Highlighting}[]
\NormalTok{normu <-}\StringTok{ }\KeywordTok{sqrt}\NormalTok{(u[}\DecValTok{1}\NormalTok{]}\OperatorTok{^}\DecValTok{2} \OperatorTok{+}\StringTok{ }\NormalTok{u[}\DecValTok{2}\NormalTok{]}\OperatorTok{^}\DecValTok{2}\NormalTok{)}
\NormalTok{normv <-}\StringTok{ }\KeywordTok{sqrt}\NormalTok{(v[}\DecValTok{1}\NormalTok{]}\OperatorTok{^}\DecValTok{2} \OperatorTok{+}\StringTok{ }\NormalTok{v[}\DecValTok{2}\NormalTok{]}\OperatorTok{^}\DecValTok{2}\NormalTok{)}
\NormalTok{normu}
\end{Highlighting}
\end{Shaded}

\begin{verbatim}
## [1] 0.7071068
\end{verbatim}

\begin{Shaded}
\begin{Highlighting}[]
\NormalTok{normv}
\end{Highlighting}
\end{Shaded}

\begin{verbatim}
## [1] 5
\end{verbatim}

\subsection{3. What is the linear combination
3u-2v?}\label{what-is-the-linear-combination-3u-2v}

\subsubsection{Solution}\label{solution-2}

\[ a\textbf{u} + b\textbf{v} = [a\cdot\textbf{u}_1, a\cdot\textbf{u}_2, ..., a\cdot\textbf{u}_n] + [b\cdot\textbf{v}_1, b\cdot\textbf{v}_2, ..., b\cdot\textbf{v}_n]\]

\[[3\cdot0.5, 3\cdot0.5] - [2\cdot3, 2\cdot-4] = [1.5, 1.5] - [6, -8] = [-4.5, 9.5]\].

\begin{Shaded}
\begin{Highlighting}[]
\KeywordTok{c}\NormalTok{(}\DecValTok{3}\OperatorTok{*}\NormalTok{u }\OperatorTok{-}\StringTok{ }\DecValTok{2}\OperatorTok{*}\NormalTok{v)}
\end{Highlighting}
\end{Shaded}

\begin{verbatim}
## [1] -4.5  9.5
\end{verbatim}

\subsection{4. What is the angle between u and
v?}\label{what-is-the-angle-between-u-and-v}

\subsubsection{Solution}\label{solution-3}

\[\cos(\theta) = \frac{\textbf{u}\cdot\textbf{v}}{\textbf{||u|| ||v||}} \ \ \to \ \ \theta = cos^{-1}\Bigg(\frac{\textbf{u}\cdot\textbf{v}}{\textbf{||u|| ||v||}}\Bigg) \]
\[\theta = cos^{-1}\Bigg(\frac{-0.5}{0.7071*5}\Bigg) = cos^{-1}(-0.1414) \approx 1.7127 \text{ rad} \approx 98.13^{\circ}.\]

\section{Problem Set 2}\label{problem-set-2}

\begin{Shaded}
\begin{Highlighting}[]
\NormalTok{matsolver <-}\StringTok{ }\ControlFlowTok{function}\NormalTok{(a, b)\{}
  \CommentTok{# Assume a is the matrix, and b is the constrain. }
  \CommentTok{# Merge the two into an augmented matrix}
\NormalTok{  augmm <-}\StringTok{ }\KeywordTok{cbind}\NormalTok{(a, b)}
  
  \CommentTok{# Need a 1 in position [1,1]}
  \ControlFlowTok{if}\NormalTok{ (augmm[}\DecValTok{1}\NormalTok{,}\DecValTok{1}\NormalTok{] }\OperatorTok{==}\StringTok{ }\DecValTok{0}\NormalTok{)\{}
    \ControlFlowTok{if}\NormalTok{ (augmm[}\DecValTok{2}\NormalTok{,}\DecValTok{1}\NormalTok{] }\OperatorTok{!=}\StringTok{ }\DecValTok{0}\NormalTok{)\{}
\NormalTok{      augmm <-}\StringTok{ }\NormalTok{augmm[}\KeywordTok{c}\NormalTok{(}\DecValTok{2}\NormalTok{,}\DecValTok{1}\NormalTok{,}\DecValTok{3}\NormalTok{),]}
\NormalTok{    \}}
    \ControlFlowTok{else}\NormalTok{ \{}
\NormalTok{      augmm <-}\StringTok{ }\NormalTok{augmm[}\KeywordTok{c}\NormalTok{(}\DecValTok{3}\NormalTok{,}\DecValTok{2}\NormalTok{,}\DecValTok{1}\NormalTok{),]}
\NormalTok{    \}}
\NormalTok{  \}}
  \ControlFlowTok{else} \ControlFlowTok{if}\NormalTok{ (augmm[}\DecValTok{1}\NormalTok{,}\DecValTok{1}\NormalTok{] }\OperatorTok{!=}\StringTok{ }\DecValTok{1}\NormalTok{)\{}
\NormalTok{    augmm[}\DecValTok{1}\NormalTok{,] <-}\StringTok{ }\NormalTok{augmm[}\DecValTok{1}\NormalTok{,] }\OperatorTok{/}\StringTok{ }\NormalTok{augmm[}\DecValTok{1}\NormalTok{,}\DecValTok{1}\NormalTok{] }
\NormalTok{  \}}
  
  \CommentTok{# All other rows in column 1 should be zero}
  \ControlFlowTok{if}\NormalTok{ (augmm[}\DecValTok{2}\NormalTok{,}\DecValTok{1}\NormalTok{] }\OperatorTok{!=}\StringTok{ }\DecValTok{0}\NormalTok{)\{}
\NormalTok{    augmm[}\DecValTok{2}\NormalTok{,] <-}\StringTok{ }\NormalTok{(augmm[}\DecValTok{2}\NormalTok{,}\DecValTok{1}\NormalTok{] }\OperatorTok{*}\StringTok{ }\NormalTok{augmm[}\DecValTok{1}\NormalTok{,]) }\OperatorTok{-}\StringTok{ }\NormalTok{augmm[}\DecValTok{2}\NormalTok{,]}
\NormalTok{  \}}
  \ControlFlowTok{if}\NormalTok{ (augmm[}\DecValTok{3}\NormalTok{,}\DecValTok{1}\NormalTok{] }\OperatorTok{!=}\StringTok{ }\DecValTok{0}\NormalTok{)\{}
\NormalTok{    augmm[}\DecValTok{3}\NormalTok{,] <-}\StringTok{ }\NormalTok{(augmm[}\DecValTok{3}\NormalTok{,}\DecValTok{1}\NormalTok{] }\OperatorTok{*}\StringTok{ }\NormalTok{augmm[}\DecValTok{1}\NormalTok{,]) }\OperatorTok{-}\StringTok{ }\NormalTok{augmm[}\DecValTok{3}\NormalTok{,]}
\NormalTok{  \}}
  
  \CommentTok{#Need a non-zero number in the middle}
  \ControlFlowTok{if}\NormalTok{ (augmm[}\DecValTok{2}\NormalTok{,}\DecValTok{2}\NormalTok{] }\OperatorTok{==}\StringTok{ }\DecValTok{0}\NormalTok{)\{}
\NormalTok{    augmm <-}\StringTok{ }\NormalTok{augmm[}\KeywordTok{c}\NormalTok{(}\DecValTok{1}\NormalTok{,}\DecValTok{3}\NormalTok{,}\DecValTok{2}\NormalTok{),]}
\NormalTok{  \}}
  
\NormalTok{  augmm[}\DecValTok{2}\NormalTok{,] <-}\StringTok{ }\NormalTok{augmm[}\DecValTok{2}\NormalTok{,] }\OperatorTok{/}\StringTok{ }\NormalTok{augmm[}\DecValTok{2}\NormalTok{,}\DecValTok{2}\NormalTok{]}
\NormalTok{  augmm[}\DecValTok{3}\NormalTok{,] <-}\StringTok{ }\NormalTok{(augmm[}\DecValTok{3}\NormalTok{,}\DecValTok{2}\NormalTok{] }\OperatorTok{*}\StringTok{ }\NormalTok{augmm[}\DecValTok{2}\NormalTok{,]) }\OperatorTok{-}\StringTok{ }\NormalTok{augmm[}\DecValTok{3}\NormalTok{,]}
  
  \CommentTok{# Use back substitution}
\NormalTok{  solvedm <-}\StringTok{ }\KeywordTok{numeric}\NormalTok{(}\DecValTok{3}\NormalTok{)}
\NormalTok{  solvedm[}\DecValTok{3}\NormalTok{] <-}\StringTok{ }\NormalTok{augmm[}\DecValTok{3}\NormalTok{,}\DecValTok{4}\NormalTok{] }\OperatorTok{/}\StringTok{ }\NormalTok{augmm[}\DecValTok{3}\NormalTok{,}\DecValTok{3}\NormalTok{]}
\NormalTok{  solvedm[}\DecValTok{2}\NormalTok{] <-}\StringTok{ }\NormalTok{(augmm[}\DecValTok{2}\NormalTok{,}\DecValTok{4}\NormalTok{] }\OperatorTok{-}\StringTok{ }\NormalTok{augmm[}\DecValTok{2}\NormalTok{,}\DecValTok{3}\NormalTok{] }\OperatorTok{*}\StringTok{ }\NormalTok{solvedm[}\DecValTok{3}\NormalTok{]) }\OperatorTok{/}\StringTok{ }\NormalTok{augmm[}\DecValTok{2}\NormalTok{,}\DecValTok{2}\NormalTok{]}
\NormalTok{  solvedm[}\DecValTok{1}\NormalTok{] <-}\StringTok{ }\NormalTok{(augmm[}\DecValTok{1}\NormalTok{,}\DecValTok{4}\NormalTok{] }\OperatorTok{-}\StringTok{ }\NormalTok{augmm[}\DecValTok{1}\NormalTok{,}\DecValTok{2}\NormalTok{] }\OperatorTok{*}\StringTok{ }\NormalTok{solvedm[}\DecValTok{2}\NormalTok{] }\OperatorTok{-}\StringTok{ }\NormalTok{augmm[}\DecValTok{1}\NormalTok{,}\DecValTok{3}\NormalTok{] }\OperatorTok{*}\StringTok{ }\NormalTok{solvedm[}\DecValTok{3}\NormalTok{]) }\OperatorTok{/}\StringTok{ }\NormalTok{augmm[}\DecValTok{1}\NormalTok{,}\DecValTok{1}\NormalTok{]}
  
  \KeywordTok{return}\NormalTok{(}\KeywordTok{cbind}\NormalTok{(solvedm))}
\NormalTok{\}}
\end{Highlighting}
\end{Shaded}


\end{document}
