\documentclass[]{article}
\usepackage{lmodern}
\usepackage{amssymb,amsmath}
\usepackage{ifxetex,ifluatex}
\usepackage{fixltx2e} % provides \textsubscript
\ifnum 0\ifxetex 1\fi\ifluatex 1\fi=0 % if pdftex
  \usepackage[T1]{fontenc}
  \usepackage[utf8]{inputenc}
\else % if luatex or xelatex
  \ifxetex
    \usepackage{mathspec}
  \else
    \usepackage{fontspec}
  \fi
  \defaultfontfeatures{Ligatures=TeX,Scale=MatchLowercase}
\fi
% use upquote if available, for straight quotes in verbatim environments
\IfFileExists{upquote.sty}{\usepackage{upquote}}{}
% use microtype if available
\IfFileExists{microtype.sty}{%
\usepackage{microtype}
\UseMicrotypeSet[protrusion]{basicmath} % disable protrusion for tt fonts
}{}
\usepackage[margin=1in]{geometry}
\usepackage{hyperref}
\hypersetup{unicode=true,
            pdftitle={DATA 605 - Assignment 3},
            pdfauthor={Joshua Sturm},
            pdfborder={0 0 0},
            breaklinks=true}
\urlstyle{same}  % don't use monospace font for urls
\usepackage{color}
\usepackage{fancyvrb}
\newcommand{\VerbBar}{|}
\newcommand{\VERB}{\Verb[commandchars=\\\{\}]}
\DefineVerbatimEnvironment{Highlighting}{Verbatim}{commandchars=\\\{\}}
% Add ',fontsize=\small' for more characters per line
\usepackage{framed}
\definecolor{shadecolor}{RGB}{248,248,248}
\newenvironment{Shaded}{\begin{snugshade}}{\end{snugshade}}
\newcommand{\KeywordTok}[1]{\textcolor[rgb]{0.13,0.29,0.53}{\textbf{#1}}}
\newcommand{\DataTypeTok}[1]{\textcolor[rgb]{0.13,0.29,0.53}{#1}}
\newcommand{\DecValTok}[1]{\textcolor[rgb]{0.00,0.00,0.81}{#1}}
\newcommand{\BaseNTok}[1]{\textcolor[rgb]{0.00,0.00,0.81}{#1}}
\newcommand{\FloatTok}[1]{\textcolor[rgb]{0.00,0.00,0.81}{#1}}
\newcommand{\ConstantTok}[1]{\textcolor[rgb]{0.00,0.00,0.00}{#1}}
\newcommand{\CharTok}[1]{\textcolor[rgb]{0.31,0.60,0.02}{#1}}
\newcommand{\SpecialCharTok}[1]{\textcolor[rgb]{0.00,0.00,0.00}{#1}}
\newcommand{\StringTok}[1]{\textcolor[rgb]{0.31,0.60,0.02}{#1}}
\newcommand{\VerbatimStringTok}[1]{\textcolor[rgb]{0.31,0.60,0.02}{#1}}
\newcommand{\SpecialStringTok}[1]{\textcolor[rgb]{0.31,0.60,0.02}{#1}}
\newcommand{\ImportTok}[1]{#1}
\newcommand{\CommentTok}[1]{\textcolor[rgb]{0.56,0.35,0.01}{\textit{#1}}}
\newcommand{\DocumentationTok}[1]{\textcolor[rgb]{0.56,0.35,0.01}{\textbf{\textit{#1}}}}
\newcommand{\AnnotationTok}[1]{\textcolor[rgb]{0.56,0.35,0.01}{\textbf{\textit{#1}}}}
\newcommand{\CommentVarTok}[1]{\textcolor[rgb]{0.56,0.35,0.01}{\textbf{\textit{#1}}}}
\newcommand{\OtherTok}[1]{\textcolor[rgb]{0.56,0.35,0.01}{#1}}
\newcommand{\FunctionTok}[1]{\textcolor[rgb]{0.00,0.00,0.00}{#1}}
\newcommand{\VariableTok}[1]{\textcolor[rgb]{0.00,0.00,0.00}{#1}}
\newcommand{\ControlFlowTok}[1]{\textcolor[rgb]{0.13,0.29,0.53}{\textbf{#1}}}
\newcommand{\OperatorTok}[1]{\textcolor[rgb]{0.81,0.36,0.00}{\textbf{#1}}}
\newcommand{\BuiltInTok}[1]{#1}
\newcommand{\ExtensionTok}[1]{#1}
\newcommand{\PreprocessorTok}[1]{\textcolor[rgb]{0.56,0.35,0.01}{\textit{#1}}}
\newcommand{\AttributeTok}[1]{\textcolor[rgb]{0.77,0.63,0.00}{#1}}
\newcommand{\RegionMarkerTok}[1]{#1}
\newcommand{\InformationTok}[1]{\textcolor[rgb]{0.56,0.35,0.01}{\textbf{\textit{#1}}}}
\newcommand{\WarningTok}[1]{\textcolor[rgb]{0.56,0.35,0.01}{\textbf{\textit{#1}}}}
\newcommand{\AlertTok}[1]{\textcolor[rgb]{0.94,0.16,0.16}{#1}}
\newcommand{\ErrorTok}[1]{\textcolor[rgb]{0.64,0.00,0.00}{\textbf{#1}}}
\newcommand{\NormalTok}[1]{#1}
\usepackage{graphicx,grffile}
\makeatletter
\def\maxwidth{\ifdim\Gin@nat@width>\linewidth\linewidth\else\Gin@nat@width\fi}
\def\maxheight{\ifdim\Gin@nat@height>\textheight\textheight\else\Gin@nat@height\fi}
\makeatother
% Scale images if necessary, so that they will not overflow the page
% margins by default, and it is still possible to overwrite the defaults
% using explicit options in \includegraphics[width, height, ...]{}
\setkeys{Gin}{width=\maxwidth,height=\maxheight,keepaspectratio}
\IfFileExists{parskip.sty}{%
\usepackage{parskip}
}{% else
\setlength{\parindent}{0pt}
\setlength{\parskip}{6pt plus 2pt minus 1pt}
}
\setlength{\emergencystretch}{3em}  % prevent overfull lines
\providecommand{\tightlist}{%
  \setlength{\itemsep}{0pt}\setlength{\parskip}{0pt}}
\setcounter{secnumdepth}{0}
% Redefines (sub)paragraphs to behave more like sections
\ifx\paragraph\undefined\else
\let\oldparagraph\paragraph
\renewcommand{\paragraph}[1]{\oldparagraph{#1}\mbox{}}
\fi
\ifx\subparagraph\undefined\else
\let\oldsubparagraph\subparagraph
\renewcommand{\subparagraph}[1]{\oldsubparagraph{#1}\mbox{}}
\fi

%%% Use protect on footnotes to avoid problems with footnotes in titles
\let\rmarkdownfootnote\footnote%
\def\footnote{\protect\rmarkdownfootnote}

%%% Change title format to be more compact
\usepackage{titling}

% Create subtitle command for use in maketitle
\newcommand{\subtitle}[1]{
  \posttitle{
    \begin{center}\large#1\end{center}
    }
}

\setlength{\droptitle}{-2em}
  \title{DATA 605 - Assignment 3}
  \pretitle{\vspace{\droptitle}\centering\huge}
  \posttitle{\par}
  \author{Joshua Sturm}
  \preauthor{\centering\large\emph}
  \postauthor{\par}
  \predate{\centering\large\emph}
  \postdate{\par}
  \date{02/18/2018}


\begin{document}
\maketitle

\begin{Shaded}
\begin{Highlighting}[]
\KeywordTok{library}\NormalTok{(pracma)}
\end{Highlighting}
\end{Shaded}

\section{Problem Set 1}\label{problem-set-1}

\subsection{\texorpdfstring{1. What is the rank of matrix
\(A\)?}{1. What is the rank of matrix A?}}\label{what-is-the-rank-of-matrix-a}

\[A = \begin{bmatrix} 1 & 2 & 3 & 4\\
                     -1 & 0 & 1 & 3\\
                     0 & 1 & -2 & 1 \\
                     5 & 4 & -2 & -3
     \end{bmatrix}\]

\begin{Shaded}
\begin{Highlighting}[]
\NormalTok{A <-}\StringTok{ }\KeywordTok{matrix}\NormalTok{(}\KeywordTok{c}\NormalTok{(}\DecValTok{1}\NormalTok{, }\DecValTok{2}\NormalTok{, }\DecValTok{3}\NormalTok{, }\DecValTok{4}\NormalTok{,}
             \OperatorTok{-}\DecValTok{1}\NormalTok{, }\DecValTok{0}\NormalTok{, }\DecValTok{1}\NormalTok{, }\DecValTok{3}\NormalTok{,}
             \DecValTok{0}\NormalTok{, }\DecValTok{1}\NormalTok{, }\OperatorTok{-}\DecValTok{2}\NormalTok{, }\DecValTok{1}\NormalTok{,}
             \DecValTok{5}\NormalTok{, }\DecValTok{4}\NormalTok{, }\OperatorTok{-}\DecValTok{2}\NormalTok{, }\OperatorTok{-}\DecValTok{3}\NormalTok{),}
           \DataTypeTok{nrow =} \DecValTok{4}\NormalTok{, }\DataTypeTok{ncol =} \DecValTok{4}\NormalTok{, }\DataTypeTok{byrow =}\NormalTok{ T)}

\KeywordTok{rref}\NormalTok{(A)}
\NormalTok{##      [,1] [,2] [,3] [,4]}
\NormalTok{## [1,]    1    0    0    0}
\NormalTok{## [2,]    0    1    0    0}
\NormalTok{## [3,]    0    0    1    0}
\NormalTok{## [4,]    0    0    0    1}
\end{Highlighting}
\end{Shaded}

There are 4 pivot columns in the reduced row echelon form of the matrix,
so it has a rank of 4.

\subsection{\texorpdfstring{2. Given an \(m\times n\) matrix where
\(m>n\), what can be the maximum rank? The minimum rank, assuming that
the matrix is
non-zero?}{2. Given an m\textbackslash{}times n matrix where m\textgreater{}n, what can be the maximum rank? The minimum rank, assuming that the matrix is non-zero?}}\label{given-an-mtimes-n-matrix-where-mn-what-can-be-the-maximum-rank-the-minimum-rank-assuming-that-the-matrix-is-non-zero}

Since the rank is the number of linearly independent columns, the
maximum rank can't be larger than the number of columns, i.e. \(n\).

If the matrix is non-zero, the minimum rank would be 1.

\subsection{\texorpdfstring{3. What is the rank of matrix
\(B\)?}{3. What is the rank of matrix B?}}\label{what-is-the-rank-of-matrix-b}

\[B = \begin{bmatrix} 1 & 2 & 1\\
                     3 & 6 & 3\\
                     2 & 4 & 2
     \end{bmatrix}\]

Using the same process as question 1:

\begin{Shaded}
\begin{Highlighting}[]
\NormalTok{B <-}\StringTok{ }\KeywordTok{matrix}\NormalTok{(}\KeywordTok{c}\NormalTok{(}\DecValTok{1}\NormalTok{, }\DecValTok{2}\NormalTok{, }\DecValTok{1}\NormalTok{,}
              \DecValTok{3}\NormalTok{, }\DecValTok{6}\NormalTok{, }\DecValTok{3}\NormalTok{,}
              \DecValTok{2}\NormalTok{, }\DecValTok{4}\NormalTok{, }\DecValTok{2}\NormalTok{),}
            \DataTypeTok{nrow =} \DecValTok{3}\NormalTok{, }\DataTypeTok{ncol =} \DecValTok{3}\NormalTok{, }\DataTypeTok{byrow =}\NormalTok{ T)}

\KeywordTok{rref}\NormalTok{(B)}
\NormalTok{##      [,1] [,2] [,3]}
\NormalTok{## [1,]    1    2    1}
\NormalTok{## [2,]    0    0    0}
\NormalTok{## [3,]    0    0    0}
\end{Highlighting}
\end{Shaded}

Since the reduced row echelon form of \(B\) only has one pivot column,
its rank is 1.

\section{Problem Set 2}\label{problem-set-2}

\subsection{\texorpdfstring{1. Compute the eigenvalues and eigenvectors
of the matrix \(A\). You'll need to show your work. You'll need to write
out the characteristic polynomial and show your
solution.}{1. Compute the eigenvalues and eigenvectors of the matrix A. You'll need to show your work. You'll need to write out the characteristic polynomial and show your solution.}}\label{compute-the-eigenvalues-and-eigenvectors-of-the-matrix-a.-youll-need-to-show-your-work.-youll-need-to-write-out-the-characteristic-polynomial-and-show-your-solution.}

\[A = \begin{bmatrix} 1 & 2 & 3\\
                     0 & 4 & 5\\
                     0 & 0 & 6
     \end{bmatrix}\]

\[|A - \lambda I | = \left | \begin{bmatrix} 1 & 2 & 3\\ 0 & 4 & 5\\0 & 0 & 6\end{bmatrix} - \lambda \begin{bmatrix}1 & 0 & 0 \\ 0 & 1 & 0\\ 0 & 0 & 1 \end{bmatrix} \right |\]

\(= \left|\begin{bmatrix}1-\lambda & 2 & 3\\ 0 & 4-\lambda & 5\\ 0 & 0 & 6-\lambda\end{bmatrix}\right|\)

\(= (1-\lambda)(4-\lambda)(6-\lambda)\)

\((1 - \lambda)\Big[24-10\lambda+\lambda^2\Big] = 24 -10\lambda + \lambda^2 -24\lambda +10\lambda^2 - \lambda^3 = -\lambda^3 + 11\lambda^2 -34\lambda + 24\)

After factoring, we get \((\lambda - 1)(\lambda - 4)(\lambda - 6)\).

So the eigenvalues are 1, 4, 6.

For \(\lambda = 1\):

\begin{Shaded}
\begin{Highlighting}[]
\NormalTok{A <-}\StringTok{ }\KeywordTok{matrix}\NormalTok{(}\KeywordTok{c}\NormalTok{(}\DecValTok{1}\NormalTok{, }\DecValTok{2}\NormalTok{, }\DecValTok{3}\NormalTok{,}
              \DecValTok{0}\NormalTok{, }\DecValTok{4}\NormalTok{, }\DecValTok{5}\NormalTok{,}
              \DecValTok{0}\NormalTok{, }\DecValTok{0}\NormalTok{, }\DecValTok{6}\NormalTok{),}
            \DataTypeTok{nrow =} \DecValTok{3}\NormalTok{, }\DataTypeTok{ncol =} \DecValTok{3}\NormalTok{, }\DataTypeTok{byrow =}\NormalTok{ T)}

\NormalTok{I <-}\StringTok{ }\KeywordTok{diag}\NormalTok{(}\DecValTok{3}\NormalTok{)}

\KeywordTok{rref}\NormalTok{(A}\OperatorTok{-}\NormalTok{I)}
\NormalTok{##      [,1] [,2] [,3]}
\NormalTok{## [1,]    0    1    0}
\NormalTok{## [2,]    0    0    1}
\NormalTok{## [3,]    0    0    0}
\end{Highlighting}
\end{Shaded}

\(v_1 = \begin{bmatrix}1\\0\\0\\\end{bmatrix}\)

For \(\lambda = 4\):

\begin{Shaded}
\begin{Highlighting}[]
\KeywordTok{rref}\NormalTok{(A }\OperatorTok{-}\StringTok{ }\DecValTok{4}\OperatorTok{*}\NormalTok{I)}
\NormalTok{##      [,1]       [,2] [,3]}
\NormalTok{## [1,]    1 -0.6666667    0}
\NormalTok{## [2,]    0  0.0000000    1}
\NormalTok{## [3,]    0  0.0000000    0}
\end{Highlighting}
\end{Shaded}

\(x_1 - \frac{2}{3}x_2 = 0\)

\(v_4 = \begin{bmatrix}\frac{2}{3}\\1\\0\end{bmatrix}\)

For \(\lambda = 6\):

\begin{Shaded}
\begin{Highlighting}[]
\KeywordTok{rref}\NormalTok{(A }\OperatorTok{-}\StringTok{ }\DecValTok{6}\OperatorTok{*}\NormalTok{I)}
\NormalTok{##      [,1] [,2] [,3]}
\NormalTok{## [1,]    1    0 -1.6}
\NormalTok{## [2,]    0    1 -2.5}
\NormalTok{## [3,]    0    0  0.0}
\end{Highlighting}
\end{Shaded}

\(x_1 - 1.6x_3 = 0 \hspace{20mm} x_2 - 2.5x_3 = 0 \hspace{5mm} x_3\neq 0\).

\(v_6 = \begin{bmatrix}1.6 \\ 2.5 \\ 1 \end{bmatrix}\)


\end{document}
