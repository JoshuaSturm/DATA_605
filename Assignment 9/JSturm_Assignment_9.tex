\documentclass[]{article}
\usepackage{lmodern}
\usepackage{amssymb,amsmath}
\usepackage{ifxetex,ifluatex}
\usepackage{fixltx2e} % provides \textsubscript
\ifnum 0\ifxetex 1\fi\ifluatex 1\fi=0 % if pdftex
  \usepackage[T1]{fontenc}
  \usepackage[utf8]{inputenc}
\else % if luatex or xelatex
  \ifxetex
    \usepackage{mathspec}
  \else
    \usepackage{fontspec}
  \fi
  \defaultfontfeatures{Ligatures=TeX,Scale=MatchLowercase}
\fi
% use upquote if available, for straight quotes in verbatim environments
\IfFileExists{upquote.sty}{\usepackage{upquote}}{}
% use microtype if available
\IfFileExists{microtype.sty}{%
\usepackage{microtype}
\UseMicrotypeSet[protrusion]{basicmath} % disable protrusion for tt fonts
}{}
\usepackage[margin=1in]{geometry}
\usepackage{hyperref}
\hypersetup{unicode=true,
            pdftitle={DATA 605 - Assignment 9},
            pdfauthor={Joshua Sturm},
            pdfborder={0 0 0},
            breaklinks=true}
\urlstyle{same}  % don't use monospace font for urls
\usepackage{color}
\usepackage{fancyvrb}
\newcommand{\VerbBar}{|}
\newcommand{\VERB}{\Verb[commandchars=\\\{\}]}
\DefineVerbatimEnvironment{Highlighting}{Verbatim}{commandchars=\\\{\}}
% Add ',fontsize=\small' for more characters per line
\usepackage{framed}
\definecolor{shadecolor}{RGB}{248,248,248}
\newenvironment{Shaded}{\begin{snugshade}}{\end{snugshade}}
\newcommand{\KeywordTok}[1]{\textcolor[rgb]{0.13,0.29,0.53}{\textbf{#1}}}
\newcommand{\DataTypeTok}[1]{\textcolor[rgb]{0.13,0.29,0.53}{#1}}
\newcommand{\DecValTok}[1]{\textcolor[rgb]{0.00,0.00,0.81}{#1}}
\newcommand{\BaseNTok}[1]{\textcolor[rgb]{0.00,0.00,0.81}{#1}}
\newcommand{\FloatTok}[1]{\textcolor[rgb]{0.00,0.00,0.81}{#1}}
\newcommand{\ConstantTok}[1]{\textcolor[rgb]{0.00,0.00,0.00}{#1}}
\newcommand{\CharTok}[1]{\textcolor[rgb]{0.31,0.60,0.02}{#1}}
\newcommand{\SpecialCharTok}[1]{\textcolor[rgb]{0.00,0.00,0.00}{#1}}
\newcommand{\StringTok}[1]{\textcolor[rgb]{0.31,0.60,0.02}{#1}}
\newcommand{\VerbatimStringTok}[1]{\textcolor[rgb]{0.31,0.60,0.02}{#1}}
\newcommand{\SpecialStringTok}[1]{\textcolor[rgb]{0.31,0.60,0.02}{#1}}
\newcommand{\ImportTok}[1]{#1}
\newcommand{\CommentTok}[1]{\textcolor[rgb]{0.56,0.35,0.01}{\textit{#1}}}
\newcommand{\DocumentationTok}[1]{\textcolor[rgb]{0.56,0.35,0.01}{\textbf{\textit{#1}}}}
\newcommand{\AnnotationTok}[1]{\textcolor[rgb]{0.56,0.35,0.01}{\textbf{\textit{#1}}}}
\newcommand{\CommentVarTok}[1]{\textcolor[rgb]{0.56,0.35,0.01}{\textbf{\textit{#1}}}}
\newcommand{\OtherTok}[1]{\textcolor[rgb]{0.56,0.35,0.01}{#1}}
\newcommand{\FunctionTok}[1]{\textcolor[rgb]{0.00,0.00,0.00}{#1}}
\newcommand{\VariableTok}[1]{\textcolor[rgb]{0.00,0.00,0.00}{#1}}
\newcommand{\ControlFlowTok}[1]{\textcolor[rgb]{0.13,0.29,0.53}{\textbf{#1}}}
\newcommand{\OperatorTok}[1]{\textcolor[rgb]{0.81,0.36,0.00}{\textbf{#1}}}
\newcommand{\BuiltInTok}[1]{#1}
\newcommand{\ExtensionTok}[1]{#1}
\newcommand{\PreprocessorTok}[1]{\textcolor[rgb]{0.56,0.35,0.01}{\textit{#1}}}
\newcommand{\AttributeTok}[1]{\textcolor[rgb]{0.77,0.63,0.00}{#1}}
\newcommand{\RegionMarkerTok}[1]{#1}
\newcommand{\InformationTok}[1]{\textcolor[rgb]{0.56,0.35,0.01}{\textbf{\textit{#1}}}}
\newcommand{\WarningTok}[1]{\textcolor[rgb]{0.56,0.35,0.01}{\textbf{\textit{#1}}}}
\newcommand{\AlertTok}[1]{\textcolor[rgb]{0.94,0.16,0.16}{#1}}
\newcommand{\ErrorTok}[1]{\textcolor[rgb]{0.64,0.00,0.00}{\textbf{#1}}}
\newcommand{\NormalTok}[1]{#1}
\usepackage{graphicx,grffile}
\makeatletter
\def\maxwidth{\ifdim\Gin@nat@width>\linewidth\linewidth\else\Gin@nat@width\fi}
\def\maxheight{\ifdim\Gin@nat@height>\textheight\textheight\else\Gin@nat@height\fi}
\makeatother
% Scale images if necessary, so that they will not overflow the page
% margins by default, and it is still possible to overwrite the defaults
% using explicit options in \includegraphics[width, height, ...]{}
\setkeys{Gin}{width=\maxwidth,height=\maxheight,keepaspectratio}
\IfFileExists{parskip.sty}{%
\usepackage{parskip}
}{% else
\setlength{\parindent}{0pt}
\setlength{\parskip}{6pt plus 2pt minus 1pt}
}
\setlength{\emergencystretch}{3em}  % prevent overfull lines
\providecommand{\tightlist}{%
  \setlength{\itemsep}{0pt}\setlength{\parskip}{0pt}}
\setcounter{secnumdepth}{0}
% Redefines (sub)paragraphs to behave more like sections
\ifx\paragraph\undefined\else
\let\oldparagraph\paragraph
\renewcommand{\paragraph}[1]{\oldparagraph{#1}\mbox{}}
\fi
\ifx\subparagraph\undefined\else
\let\oldsubparagraph\subparagraph
\renewcommand{\subparagraph}[1]{\oldsubparagraph{#1}\mbox{}}
\fi

%%% Use protect on footnotes to avoid problems with footnotes in titles
\let\rmarkdownfootnote\footnote%
\def\footnote{\protect\rmarkdownfootnote}

%%% Change title format to be more compact
\usepackage{titling}

% Create subtitle command for use in maketitle
\newcommand{\subtitle}[1]{
  \posttitle{
    \begin{center}\large#1\end{center}
    }
}

\setlength{\droptitle}{-2em}
  \title{DATA 605 - Assignment 9}
  \pretitle{\vspace{\droptitle}\centering\huge}
  \posttitle{\par}
  \author{Joshua Sturm}
  \preauthor{\centering\large\emph}
  \postauthor{\par}
  \predate{\centering\large\emph}
  \postdate{\par}
  \date{03/29/2018}


\begin{document}
\maketitle

\section{Question 1 (9.3 \#11)}\label{question-1-9.3-11}

The price of one share of stock in the Pilsdorff Beer Company (see
Exercise 8.2.12) is given by \(Y_n\) on the \(n\)th day of the year.
Finn observes that the differences \(X_n = Y_{n+1} - Y_n\) appear to be
independent random variables with a common distribution having mean
\(\mu = 0\) and variance \(\sigma^2 = \frac{1}{4}\). If \(Y_1 = 100\),
estimate the probability that \(Y_{365}\) is:

\begin{enumerate}
\def\labelenumi{(\alph{enumi})}
\item
  \(\geq 100\).
\item
  \(\geq 110\).
\item
  \(\geq 120\).
\end{enumerate}

\subsection{Solution}\label{solution}

\(Y_{365} = Y_1 + X_1 + X_2 + ... + X_{364}.\)

\(S_n = Y_{n+1} - 100\).

If \(n = 364, S_{364} = Y_{365} - 100 \ \to \ Y_{365} = S_{364} + 100\).

\(E[S_{364}] = n\mu = 364\cdot 0 = 0\).

Variance of \(S_{364} = 364\cdot\frac{1}{4} =\) 91
\(\ \to \ \sigma = \sqrt{91}\).

\subsubsection{(a)}\label{a}

\(P(Y_{365} \geq 100) = P(S_{364} + 100 \geq 100) = P(S_{364} \geq 0)\).

\begin{Shaded}
\begin{Highlighting}[]
\NormalTok{q <-}\StringTok{ }\DecValTok{0}
\NormalTok{mu <-}\StringTok{ }\DecValTok{0}
\NormalTok{sd <-}\StringTok{ }\KeywordTok{sqrt}\NormalTok{(}\DecValTok{91}\NormalTok{)}

\KeywordTok{pnorm}\NormalTok{(q, }\DataTypeTok{mean =}\NormalTok{ mu, }\DataTypeTok{sd =}\NormalTok{ sd, }\DataTypeTok{lower.tail =}\NormalTok{ F)}
\NormalTok{## [1] 0.5}
\end{Highlighting}
\end{Shaded}

\subsubsection{(b)}\label{b}

\(P(Y_{365} \geq 110) = P(S_{364} + 100 \geq 110) = P(S_{364} \geq 10)\).

\begin{Shaded}
\begin{Highlighting}[]
\NormalTok{q <-}\StringTok{ }\DecValTok{10}

\KeywordTok{pnorm}\NormalTok{(q, }\DataTypeTok{mean =}\NormalTok{ mu, }\DataTypeTok{sd =}\NormalTok{ sd, }\DataTypeTok{lower.tail =}\NormalTok{ F)}
\NormalTok{## [1] 0.1472537}
\end{Highlighting}
\end{Shaded}

\subsubsection{(c)}\label{c}

\(P(Y_{365} \geq 120) = P(S_{364} + 100 \geq 120) = P(S_{364} \geq 20)\).

\begin{Shaded}
\begin{Highlighting}[]
\NormalTok{q <-}\StringTok{ }\DecValTok{20}

\KeywordTok{pnorm}\NormalTok{(q, }\DataTypeTok{mean =}\NormalTok{ mu, }\DataTypeTok{sd =}\NormalTok{ sd, }\DataTypeTok{lower.tail =}\NormalTok{ F)}
\NormalTok{## [1] 0.01801584}
\end{Highlighting}
\end{Shaded}

\section{Question 2}\label{question-2}

Calculate the expected value and variance of the binomial distribution
using the moment generating function.

\subsection{Solution}\label{solution-1}

The pmf for the binomial distribution is \(\binom{n}{k}p^{n}q^{n-k}\).

The moment generating function of a random variable is
\(M_k(t) = E[e^{tn}], \ \ t \in \mathbb{R}\).

Plugging in the binomial formula, the moment generating function is
\[M_k(t) = \sum_{k=0}^{n}e^{tn}\binom{n}{k}p^{n}q^{n-k} = \sum_{k=0}^{n}(pe^t)^{n}\binom{n}{k}q^{n-k}\]

Simplifying: \(M_k(t) = (q + pe^t)^n\).

If we differentiate \(M_k(t)\) wrt \(t\):
\(M_{k}^{'}(t) = n(pe^t)(q+pe^t)^{n-1}\).

When \(t = 0: E[k] = np(q+p)^{n-1} = np\).

For the second moment, we take the second derivative (using the product
rule):

\(M_k^{''}(t) = np\Big[e^{t}(pe^t + q)^{n-1}+(n-1)(pe^t + q)^{n-2}(e^tp + 0)\Big]\)

Simplifying: \(M_k^{''}(t) = npe^t(pe^t + q)^{n-2}(npe^t + q)\).

When \(t = 0:\)

\(M_k^{''}(0) = E[k^2] = np(q + p)^{n-2}(q + np) = np(q + np)\).

Using the formula \(V(x) = E[x^2] - (E[x])^2:\)

\(V(k) = np(q + np) - n^2p^2 = npq\).

\section{Question 3}\label{question-3}

Calculate the expected value and variance of the exponential
distribution using the moment generating function.

\subsection{Solution}\label{solution-2}

Proceeding in the same manner as in question 2.

The pmf for the exponential distribution of a random variable is
\(\lambda e^{-\lambda x}\).

Moment generating function:
\[M_x(t) = \int_{0}^{\infty}e^{tx}\lambda e^{-\lambda x} dx = \lambda \int_{0}^{\infty}e^{(t-\lambda)x} dx = \frac{\lambda}{t - \lambda}, \ \ |t| < \lambda\]

\(M_{x}^{'}(t) = \frac{\lambda}{(\lambda - t)^2}\).

When \(t = 0:\)

\(E[X] = M_{x}^{'}(0) = \frac{1}{\lambda}\).

Second moment = second derivative:

\(E[X^2] = M_{x}^{''}(0) = \frac{2\lambda}{(\lambda - t)^3} = \frac{2}{\lambda^2}\).

Using the formula \(V(x) = E[X^2] - (E[X])^2:\)

\(V(x) = \frac{2}{\lambda^2} - \frac{1}{\lambda^2} = \frac{1}{\lambda^2}\).


\end{document}
