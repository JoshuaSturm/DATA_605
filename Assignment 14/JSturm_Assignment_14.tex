\documentclass[]{article}
\usepackage{lmodern}
\usepackage{amssymb,amsmath}
\usepackage{ifxetex,ifluatex}
\usepackage{fixltx2e} % provides \textsubscript
\ifnum 0\ifxetex 1\fi\ifluatex 1\fi=0 % if pdftex
  \usepackage[T1]{fontenc}
  \usepackage[utf8]{inputenc}
\else % if luatex or xelatex
  \ifxetex
    \usepackage{mathspec}
  \else
    \usepackage{fontspec}
  \fi
  \defaultfontfeatures{Ligatures=TeX,Scale=MatchLowercase}
\fi
% use upquote if available, for straight quotes in verbatim environments
\IfFileExists{upquote.sty}{\usepackage{upquote}}{}
% use microtype if available
\IfFileExists{microtype.sty}{%
\usepackage{microtype}
\UseMicrotypeSet[protrusion]{basicmath} % disable protrusion for tt fonts
}{}
\usepackage[margin=1in]{geometry}
\usepackage{hyperref}
\hypersetup{unicode=true,
            pdftitle={DATA\_605\_Assignment\_14},
            pdfauthor={Joshua Sturm},
            pdfborder={0 0 0},
            breaklinks=true}
\urlstyle{same}  % don't use monospace font for urls
\usepackage{graphicx,grffile}
\makeatletter
\def\maxwidth{\ifdim\Gin@nat@width>\linewidth\linewidth\else\Gin@nat@width\fi}
\def\maxheight{\ifdim\Gin@nat@height>\textheight\textheight\else\Gin@nat@height\fi}
\makeatother
% Scale images if necessary, so that they will not overflow the page
% margins by default, and it is still possible to overwrite the defaults
% using explicit options in \includegraphics[width, height, ...]{}
\setkeys{Gin}{width=\maxwidth,height=\maxheight,keepaspectratio}
\IfFileExists{parskip.sty}{%
\usepackage{parskip}
}{% else
\setlength{\parindent}{0pt}
\setlength{\parskip}{6pt plus 2pt minus 1pt}
}
\setlength{\emergencystretch}{3em}  % prevent overfull lines
\providecommand{\tightlist}{%
  \setlength{\itemsep}{0pt}\setlength{\parskip}{0pt}}
\setcounter{secnumdepth}{0}
% Redefines (sub)paragraphs to behave more like sections
\ifx\paragraph\undefined\else
\let\oldparagraph\paragraph
\renewcommand{\paragraph}[1]{\oldparagraph{#1}\mbox{}}
\fi
\ifx\subparagraph\undefined\else
\let\oldsubparagraph\subparagraph
\renewcommand{\subparagraph}[1]{\oldsubparagraph{#1}\mbox{}}
\fi

%%% Use protect on footnotes to avoid problems with footnotes in titles
\let\rmarkdownfootnote\footnote%
\def\footnote{\protect\rmarkdownfootnote}

%%% Change title format to be more compact
\usepackage{titling}

% Create subtitle command for use in maketitle
\newcommand{\subtitle}[1]{
  \posttitle{
    \begin{center}\large#1\end{center}
    }
}

\setlength{\droptitle}{-2em}
  \title{DATA\_605\_Assignment\_14}
  \pretitle{\vspace{\droptitle}\centering\huge}
  \posttitle{\par}
  \author{Joshua Sturm}
  \preauthor{\centering\large\emph}
  \postauthor{\par}
  \predate{\centering\large\emph}
  \postdate{\par}
  \date{May 13, 2018}


\begin{document}
\maketitle

\hypertarget{task}{%
\section{Task}\label{task}}

This week, we'll work out some Taylor Series expansions of popular
functions.

\[
f(x) = \frac{1}{(1-x)}
\]

\[
f(x) = e^x
\]

\[
f(x) = \ln(1 + x)
\]

For each function, only consider its valid ranges as indicated in the
notes when you are computing the Taylor Series expansion. Please submit
your assignment as a R-Markdown document.

\hypertarget{section}{%
\section{1.}\label{section}}

The formula for a Taylor Series centered about \(c\) is \begin{gather*}
f(x) = f(c) + f'(c)(x-c) + \frac{f''(c)}{2!} + \cdot\cdot\cdot \\
= \sum_{n=0}^{\infty}\frac{f^{(n)}(c)}{n!}x^n
\end{gather*}

If this is centered about 0, aka a Maclaurin Series, we get:

\begin{gather*}
1 + x + x^2 + x^3 + \cdot\cdot\cdot \\
= \sum_{n = 0}^{\infty}x^n \ \ \ x \in (-1, 1)
\end{gather*}

This is basically a geometric series.

\hypertarget{section-1}{%
\section{2.}\label{section-1}}

Centered about 0:

\begin{gather*}
1 + x + \frac{x^2}{2!} + \frac{x^3}{3!} + \cdot\cdot\cdot \\
= \sum_{n = 0}^{\infty} \frac{x^n}{n!} \ \ \ x \in \mathbb{R}
\end{gather*}

\hypertarget{section-2}{%
\section{3.}\label{section-2}}

Centered about 0:

\begin{gather*}
0 + x - \frac{1}{2!}x^2 + \frac{2}{3!}x^3 - \frac{6}{4!}x^4 + \cdot\cdot\cdot \\
= \sum_{n = 0}^{\infty}(-1)^{n+1}\frac{x^n}{n} = \sum_{n = 0}^{\infty}(-1)^{n-1}\frac{x^n}{n} \ \ \ x \in (-1, 1]
\end{gather*}


\end{document}
