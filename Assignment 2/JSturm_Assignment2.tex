\documentclass[]{article}
\usepackage{lmodern}
\usepackage{amssymb,amsmath}
\usepackage{ifxetex,ifluatex}
\usepackage{fixltx2e} % provides \textsubscript
\ifnum 0\ifxetex 1\fi\ifluatex 1\fi=0 % if pdftex
  \usepackage[T1]{fontenc}
  \usepackage[utf8]{inputenc}
\else % if luatex or xelatex
  \ifxetex
    \usepackage{mathspec}
  \else
    \usepackage{fontspec}
  \fi
  \defaultfontfeatures{Ligatures=TeX,Scale=MatchLowercase}
\fi
% use upquote if available, for straight quotes in verbatim environments
\IfFileExists{upquote.sty}{\usepackage{upquote}}{}
% use microtype if available
\IfFileExists{microtype.sty}{%
\usepackage{microtype}
\UseMicrotypeSet[protrusion]{basicmath} % disable protrusion for tt fonts
}{}
\usepackage[margin=1in]{geometry}
\usepackage{hyperref}
\hypersetup{unicode=true,
            pdftitle={DATA 605 - Assignment 2},
            pdfauthor={Joshua Sturm},
            pdfborder={0 0 0},
            breaklinks=true}
\urlstyle{same}  % don't use monospace font for urls
\usepackage{color}
\usepackage{fancyvrb}
\newcommand{\VerbBar}{|}
\newcommand{\VERB}{\Verb[commandchars=\\\{\}]}
\DefineVerbatimEnvironment{Highlighting}{Verbatim}{commandchars=\\\{\}}
% Add ',fontsize=\small' for more characters per line
\usepackage{framed}
\definecolor{shadecolor}{RGB}{248,248,248}
\newenvironment{Shaded}{\begin{snugshade}}{\end{snugshade}}
\newcommand{\KeywordTok}[1]{\textcolor[rgb]{0.13,0.29,0.53}{\textbf{#1}}}
\newcommand{\DataTypeTok}[1]{\textcolor[rgb]{0.13,0.29,0.53}{#1}}
\newcommand{\DecValTok}[1]{\textcolor[rgb]{0.00,0.00,0.81}{#1}}
\newcommand{\BaseNTok}[1]{\textcolor[rgb]{0.00,0.00,0.81}{#1}}
\newcommand{\FloatTok}[1]{\textcolor[rgb]{0.00,0.00,0.81}{#1}}
\newcommand{\ConstantTok}[1]{\textcolor[rgb]{0.00,0.00,0.00}{#1}}
\newcommand{\CharTok}[1]{\textcolor[rgb]{0.31,0.60,0.02}{#1}}
\newcommand{\SpecialCharTok}[1]{\textcolor[rgb]{0.00,0.00,0.00}{#1}}
\newcommand{\StringTok}[1]{\textcolor[rgb]{0.31,0.60,0.02}{#1}}
\newcommand{\VerbatimStringTok}[1]{\textcolor[rgb]{0.31,0.60,0.02}{#1}}
\newcommand{\SpecialStringTok}[1]{\textcolor[rgb]{0.31,0.60,0.02}{#1}}
\newcommand{\ImportTok}[1]{#1}
\newcommand{\CommentTok}[1]{\textcolor[rgb]{0.56,0.35,0.01}{\textit{#1}}}
\newcommand{\DocumentationTok}[1]{\textcolor[rgb]{0.56,0.35,0.01}{\textbf{\textit{#1}}}}
\newcommand{\AnnotationTok}[1]{\textcolor[rgb]{0.56,0.35,0.01}{\textbf{\textit{#1}}}}
\newcommand{\CommentVarTok}[1]{\textcolor[rgb]{0.56,0.35,0.01}{\textbf{\textit{#1}}}}
\newcommand{\OtherTok}[1]{\textcolor[rgb]{0.56,0.35,0.01}{#1}}
\newcommand{\FunctionTok}[1]{\textcolor[rgb]{0.00,0.00,0.00}{#1}}
\newcommand{\VariableTok}[1]{\textcolor[rgb]{0.00,0.00,0.00}{#1}}
\newcommand{\ControlFlowTok}[1]{\textcolor[rgb]{0.13,0.29,0.53}{\textbf{#1}}}
\newcommand{\OperatorTok}[1]{\textcolor[rgb]{0.81,0.36,0.00}{\textbf{#1}}}
\newcommand{\BuiltInTok}[1]{#1}
\newcommand{\ExtensionTok}[1]{#1}
\newcommand{\PreprocessorTok}[1]{\textcolor[rgb]{0.56,0.35,0.01}{\textit{#1}}}
\newcommand{\AttributeTok}[1]{\textcolor[rgb]{0.77,0.63,0.00}{#1}}
\newcommand{\RegionMarkerTok}[1]{#1}
\newcommand{\InformationTok}[1]{\textcolor[rgb]{0.56,0.35,0.01}{\textbf{\textit{#1}}}}
\newcommand{\WarningTok}[1]{\textcolor[rgb]{0.56,0.35,0.01}{\textbf{\textit{#1}}}}
\newcommand{\AlertTok}[1]{\textcolor[rgb]{0.94,0.16,0.16}{#1}}
\newcommand{\ErrorTok}[1]{\textcolor[rgb]{0.64,0.00,0.00}{\textbf{#1}}}
\newcommand{\NormalTok}[1]{#1}
\usepackage{graphicx,grffile}
\makeatletter
\def\maxwidth{\ifdim\Gin@nat@width>\linewidth\linewidth\else\Gin@nat@width\fi}
\def\maxheight{\ifdim\Gin@nat@height>\textheight\textheight\else\Gin@nat@height\fi}
\makeatother
% Scale images if necessary, so that they will not overflow the page
% margins by default, and it is still possible to overwrite the defaults
% using explicit options in \includegraphics[width, height, ...]{}
\setkeys{Gin}{width=\maxwidth,height=\maxheight,keepaspectratio}
\IfFileExists{parskip.sty}{%
\usepackage{parskip}
}{% else
\setlength{\parindent}{0pt}
\setlength{\parskip}{6pt plus 2pt minus 1pt}
}
\setlength{\emergencystretch}{3em}  % prevent overfull lines
\providecommand{\tightlist}{%
  \setlength{\itemsep}{0pt}\setlength{\parskip}{0pt}}
\setcounter{secnumdepth}{0}
% Redefines (sub)paragraphs to behave more like sections
\ifx\paragraph\undefined\else
\let\oldparagraph\paragraph
\renewcommand{\paragraph}[1]{\oldparagraph{#1}\mbox{}}
\fi
\ifx\subparagraph\undefined\else
\let\oldsubparagraph\subparagraph
\renewcommand{\subparagraph}[1]{\oldsubparagraph{#1}\mbox{}}
\fi

%%% Use protect on footnotes to avoid problems with footnotes in titles
\let\rmarkdownfootnote\footnote%
\def\footnote{\protect\rmarkdownfootnote}

%%% Change title format to be more compact
\usepackage{titling}

% Create subtitle command for use in maketitle
\newcommand{\subtitle}[1]{
  \posttitle{
    \begin{center}\large#1\end{center}
    }
}

\setlength{\droptitle}{-2em}
  \title{DATA 605 - Assignment 2}
  \pretitle{\vspace{\droptitle}\centering\huge}
  \posttitle{\par}
  \author{Joshua Sturm}
  \preauthor{\centering\large\emph}
  \postauthor{\par}
  \predate{\centering\large\emph}
  \postdate{\par}
  \date{02/05/2018}


\begin{document}
\maketitle

\section{Problem Set 1}\label{problem-set-1}

\subsection{\texorpdfstring{1. Show that \(A^TA \neq AA^T\) in general.
(Proof and
demonstration.)}{1. Show that A\^{}TA \textbackslash{}neq AA\^{}T in general. (Proof and demonstration.)}}\label{show-that-ata-neq-aat-in-general.-proof-and-demonstration.}

\subsubsection{Solution}\label{solution}

Let \(A = \begin{bmatrix}a & b\\ c & d\end{bmatrix}.\) Then
\(A^T = \begin{bmatrix}a & c\\ b & d\end{bmatrix}\).

The product of \(AA^T\) will be:
\(\begin{bmatrix}a\times a + b\times b & a\times c + b\times d\\ c\times a + d\times b & c\times c + d \times d\end{bmatrix}\).

The product of \(A^TA\) will be:
\(\begin{bmatrix}a\times a + c\times c & a\times b + c\times d\\ a\times b + c\times d & b\times b + d \times d\end{bmatrix}\).

Thus, we can see that matrix multiplication does not have the
commutative property.

In the event that \(A\) is not a square matrix, then the proof is
trivial. Say \(B\) is an \(m\times n\) matrix, then \(B^T\) will be an
\(n\times m\) matrix. \(BB^T\) will thus be an \(m\times m\) matrix,
which is a square matrix.

\subsection{\texorpdfstring{2. For a special type of square matrix
\(A\), we get \(A^TA = AA^T\). Under what conditions could this be true?
(Hint: The identity matrix \(I\) is an example of such a
matrix.)}{2. For a special type of square matrix A, we get A\^{}TA = AA\^{}T. Under what conditions could this be true? (Hint: The identity matrix I is an example of such a matrix.)}}\label{for-a-special-type-of-square-matrix-a-we-get-ata-aat.-under-what-conditions-could-this-be-true-hint-the-identity-matrix-i-is-an-example-of-such-a-matrix.}

\subsubsection{Solution}\label{solution-1}

Theorem: For any matrix \(A\), \(AA^T\) and \(A^TA\) are symmetric
matrices.\\
The product of two symmetric matrices is symmetric only if both matrices
commute, i.e. \(AA^T = A^TA\).\\
(\url{https://en.wikipedia.org/wiki/Symmetric_matrix\#Properties)/} This
is only possible when the matrix is invertible.

\section{Problem Set 3}\label{problem-set-3}

Write an R function to factorize a square matrix A into LU or LDU,
whichever you prefer. You don't have to worry about permuting rows of A
and you can assume that A is less than 5x5, if you need to hard-code any
variables in your code

\begin{Shaded}
\begin{Highlighting}[]
\CommentTok{# Assume the user inputs a square matrix.}
  
\NormalTok{factorize <-}\StringTok{ }\ControlFlowTok{function}\NormalTok{(mat)\{}
\NormalTok{  L <-}\StringTok{ }\KeywordTok{diag}\NormalTok{(}\KeywordTok{nrow}\NormalTok{(mat))    }\CommentTok{# Create lower diagonal matrix of the same size}
\NormalTok{  U <-}\StringTok{ }\NormalTok{mat}

  \CommentTok{# If the leading entry [1,1] is not > 1, swap it with a row that has a pivot}
  \ControlFlowTok{if}\NormalTok{ (U[}\DecValTok{1}\NormalTok{,}\DecValTok{1}\NormalTok{] }\OperatorTok{==}\StringTok{ }\DecValTok{0}\NormalTok{)\{}
    \ControlFlowTok{for}\NormalTok{ (i }\ControlFlowTok{in}\NormalTok{ U[}\DecValTok{2}\NormalTok{,}\DecValTok{1}\NormalTok{]}\OperatorTok{:}\NormalTok{U[nrow,}\DecValTok{1}\NormalTok{])\{}
      \ControlFlowTok{if}\NormalTok{ (U[i,}\DecValTok{1}\NormalTok{] }\OperatorTok{!=}\StringTok{ }\DecValTok{0}\NormalTok{)\{}
\NormalTok{        U[}\DecValTok{1}\NormalTok{,] <-}\StringTok{ }\NormalTok{U[i,]}
\NormalTok{      \}}
\NormalTok{    \}}
\NormalTok{  \}}
  \CommentTok{# Loop through n-1 columns, n rows, get the multipler,}
  \CommentTok{# Multiply to get the Upper Triangular matrix, and plug}
  \CommentTok{# the multiplier into the Lower Triangular Matrix}
  \ControlFlowTok{for}\NormalTok{ (k }\ControlFlowTok{in} \DecValTok{1}\OperatorTok{:}\NormalTok{(}\KeywordTok{nrow}\NormalTok{(U)}\OperatorTok{-}\DecValTok{1}\NormalTok{))\{}
    \ControlFlowTok{for}\NormalTok{ (j }\ControlFlowTok{in}\NormalTok{ (k}\OperatorTok{+}\DecValTok{1}\NormalTok{)}\OperatorTok{:}\NormalTok{(}\KeywordTok{nrow}\NormalTok{(U)))\{}
      \ControlFlowTok{if}\NormalTok{ (U[j,k] }\OperatorTok{!=}\StringTok{ }\DecValTok{0}\NormalTok{)\{}
\NormalTok{        mult <-}\StringTok{ }\NormalTok{U[j,k] }\OperatorTok{/}\StringTok{ }\NormalTok{U[k,k]}
\NormalTok{        U[j,] <-}\StringTok{ }\NormalTok{U[j,] }\OperatorTok{-}\StringTok{ }\NormalTok{(mult }\OperatorTok{*}\StringTok{ }\NormalTok{U[k,])}
\NormalTok{        L[j,k] <-}\StringTok{ }\NormalTok{mult}
\NormalTok{      \}}
\NormalTok{    \}}
\NormalTok{  \}}
  
\NormalTok{  factored.matrix <-}\StringTok{ }\KeywordTok{list}\NormalTok{(L, U)}
  
  \KeywordTok{return}\NormalTok{(factored.matrix)}
\NormalTok{\}}
\end{Highlighting}
\end{Shaded}


\end{document}
