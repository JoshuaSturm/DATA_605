\documentclass[]{article}
\usepackage{lmodern}
\usepackage{amssymb,amsmath}
\usepackage{ifxetex,ifluatex}
\usepackage{fixltx2e} % provides \textsubscript
\ifnum 0\ifxetex 1\fi\ifluatex 1\fi=0 % if pdftex
  \usepackage[T1]{fontenc}
  \usepackage[utf8]{inputenc}
\else % if luatex or xelatex
  \ifxetex
    \usepackage{mathspec}
  \else
    \usepackage{fontspec}
  \fi
  \defaultfontfeatures{Ligatures=TeX,Scale=MatchLowercase}
\fi
% use upquote if available, for straight quotes in verbatim environments
\IfFileExists{upquote.sty}{\usepackage{upquote}}{}
% use microtype if available
\IfFileExists{microtype.sty}{%
\usepackage{microtype}
\UseMicrotypeSet[protrusion]{basicmath} % disable protrusion for tt fonts
}{}
\usepackage[margin=1in]{geometry}
\usepackage{hyperref}
\hypersetup{unicode=true,
            pdftitle={DATA 605 - Assignment 5},
            pdfauthor={Joshua Sturm},
            pdfborder={0 0 0},
            breaklinks=true}
\urlstyle{same}  % don't use monospace font for urls
\usepackage{graphicx,grffile}
\makeatletter
\def\maxwidth{\ifdim\Gin@nat@width>\linewidth\linewidth\else\Gin@nat@width\fi}
\def\maxheight{\ifdim\Gin@nat@height>\textheight\textheight\else\Gin@nat@height\fi}
\makeatother
% Scale images if necessary, so that they will not overflow the page
% margins by default, and it is still possible to overwrite the defaults
% using explicit options in \includegraphics[width, height, ...]{}
\setkeys{Gin}{width=\maxwidth,height=\maxheight,keepaspectratio}
\IfFileExists{parskip.sty}{%
\usepackage{parskip}
}{% else
\setlength{\parindent}{0pt}
\setlength{\parskip}{6pt plus 2pt minus 1pt}
}
\setlength{\emergencystretch}{3em}  % prevent overfull lines
\providecommand{\tightlist}{%
  \setlength{\itemsep}{0pt}\setlength{\parskip}{0pt}}
\setcounter{secnumdepth}{0}
% Redefines (sub)paragraphs to behave more like sections
\ifx\paragraph\undefined\else
\let\oldparagraph\paragraph
\renewcommand{\paragraph}[1]{\oldparagraph{#1}\mbox{}}
\fi
\ifx\subparagraph\undefined\else
\let\oldsubparagraph\subparagraph
\renewcommand{\subparagraph}[1]{\oldsubparagraph{#1}\mbox{}}
\fi

%%% Use protect on footnotes to avoid problems with footnotes in titles
\let\rmarkdownfootnote\footnote%
\def\footnote{\protect\rmarkdownfootnote}

%%% Change title format to be more compact
\usepackage{titling}

% Create subtitle command for use in maketitle
\newcommand{\subtitle}[1]{
  \posttitle{
    \begin{center}\large#1\end{center}
    }
}

\setlength{\droptitle}{-2em}
  \title{DATA 605 - Assignment 5}
  \pretitle{\vspace{\droptitle}\centering\huge}
  \posttitle{\par}
  \author{Joshua Sturm}
  \preauthor{\centering\large\emph}
  \postauthor{\par}
  \predate{\centering\large\emph}
  \postdate{\par}
  \date{03/04/2018}


\begin{document}
\maketitle

\section{Question 1}\label{question-1}

A box contains 54 red marbles, 9 white marbles, and 75 blue marbles. If
a marble is randomly selected from the box, what is the probability that
it is red or blue? Express your answer as a fraction or a decimal number
rounded to four decimal places.

\subsection{Solution}\label{solution}

Since \(P(A)\) and \(P(B)\) are disjoint, \(P(A \cup B) = P(A) + P(B)\).

\(P(R) = \frac{54}{138}, \hspace{5mm} P(B) = \frac{75}{138}, \hspace{5mm} P(A \cup B) =\)
0.9348.

\section{Question 2}\label{question-2}

You are going to play mini golf. A ball machine that contains 19 green
golf balls, 20 red golf balls, 24 blue golf balls, and 17 yellow golf
balls, randomly gives you your ball. What is the probability that you
end up with a red golf ball? Express your answer as a simplified
fraction or a decimal rounded to four decimal places.

\subsection{Solution}\label{solution-1}

\(P(R) = \frac{20}{80}\) = 0.25.

\section{Question 3}\label{question-3}

A pizza delivery company classifies its customers by gender and location
of residence. The research department has gathered data from a random
sample of 1399 customers. The data is summarized in the table below.

\begin{center}
\begin{tabular}{ | c | c | c | }
 \hline
 \multicolumn{3}{|c|}{\textbf{Gender and Residence of Customers}} \\
 \hline
  & \textbf{Males} & \textbf{Females}\\
 \hline
 \textbf{Apartment} & 81 & 228\\
 \hline
 \textbf{Dorm} & 116 & 79\\
 \hline
 \textbf{With Parent(s)} & 215 & 252\\
 \hline
 \textbf{Sorority/Fraternity House} & 130 & 97\\
 \hline
 \textbf{Other} & 129 & 72\\
 \hline
\end{tabular}
\end{center}

What is the probability that a customer is not male or does not live
with parents? Write your answer as a fraction or a decimal number
rounded to four decimal places.

\subsection{Solution}\label{solution-2}

There are two ways to solve this problem: a simpler method, and a little
more calculative method. I'll begin with the latter.

\(P(M) = \frac{671}{1399}\) = 0.4796
~~\(\to P(M^{\complement}) = 1 - \frac{671}{1399} =\) 0.5204.

Let \(N\) be the event that the customer does not live with their
parents.

\(P(N) = 1 - P(N^{\complement}) = 1 - \frac{215 + 252}{1399} =\) 0.6662.

\(P(M^{\complement} \cap N) = \frac{228 + 79 + 97 + 72}{1399} =\)
0.3402.

Since these events are clearly not disjoint, we use a slightly modified
formula.

\(P(M^{\complement} \cup N) = P(M^{\complement}) + P(N) - P(M^{\complement} \cap N) =\)
0.8464.

The simpler method would be to subtract the opposite; that is, the
probability that the customer \(\textit{is }\) male \(\textit{and }\)
lives with their parents.

\(P(M^{\complement} \cup N) = 1 - P(M \cap N^{\complement}) = 1 - \frac{215}{1399} =\)
0.8463.

\section{Question 4}\label{question-4}

Determine if the following events are independent. - Going to the gym. -
Losing weight.

\subsection{Solution}\label{solution-3}

They're dependent, since going to the gym increases the probability of
losing weight.

\section{Question 5}\label{question-5}

A veggie wrap at City Subs is composed of 3 different vegetables and 3
different condiments wrapped up in a tortilla. If there are 8
vegetables, 7 condiments, and 3 types of tortilla available, how many
different veggie wraps can be made?

\subsection{Solution}\label{solution-4}

Since ordering doesn't matter, we can calculate it as
\(\binom{8}{3}\times\binom{7}{3}\times\binom{3}{1}\) = 5880.

\section{Question 6}\label{question-6}

Determine if the following events are independent. Jeff runs out of gas
on the way to work. Liz watches the evening news.

\subsection{Solution}\label{solution-5}

They're independent, since neither event affects the probability of the
other event happening.

\section{Question 7}\label{question-7}

The newly elected president needs to decide the remaining 8 spots
available in the cabinet he/she is appointing. If there are 14 eligible
candidates for these positions (where rank matters), how many different
ways can the members of the cabinet be appointed?

\subsection{Solution}\label{solution-6}

Since ordering matters, we're looking for the number of permutations:
\(P(14,8) = \frac{14!}{6!} = 1.2108096\times 10^{8}\).

\section{Question 8}\label{question-8}

A bag contains 9 red, 4 orange, and 9 green jellybeans. What is the
probability of reaching into the bag and randomly withdrawing 4
jellybeans such that the number of red ones is 0, the number of orange
ones is 1, and the number of green ones is 3? Write your answer as a
fraction or a decimal number rounded to four decimal places.

\subsection{Solution}\label{solution-7}

Since order doesn't matter, we're looking for a combination.

Total number of combinations: \(\binom{22}{4} =\) 7315.

The probability of picking 0 reds: \(\binom{9}{0} = 1\).

Probability of picking one orange:
\(\frac{\binom{4}{1}}{7315} = 5.468216\times 10^{-4}\).

Probability of picking three green: \(\frac{\binom{9}{3}}{7315} =\)
0.0114833.

So \(P(1O3G) = \frac{4\cdot 84}{7315} =\) 0.0459.

\section{Question 9}\label{question-9}

Evaluate the following expression: \(\frac{11!}{7!}\).

\subsection{Solution}\label{solution-8}

\(\frac{11!}{7!} = 11 \cdot 10 \cdot 9 \cdot 8 =\) 7920.

\section{Question 10}\label{question-10}

Describe the complement of the given event. 67\% of subscribers to a
fitness magazine are over the age of 34.

\subsection{Solution}\label{solution-9}

\(P(F > 34) = 0.67\)

\(P((F > 34)^{\complement}) = 1 - P(F > 34) = 1 - 0.67\) = 0.33.

33\% of subscribers to a fitness magazine are \(\leq\) the age of 34.

\section{Question 11}\label{question-11}

If you throw exactly three heads in four tosses of a coin you win \$97.
If not, you pay me \$30.

Step 1. Find the expected value of the proposition. Round your answer to
two decimal places.

Step 2. If you played this game 559 times how much would you expect to
win or lose? (Losses must be entered as negative.)

\subsection{Solution}\label{solution-10}

Let \(W\) be the event that I win. The sample space for \(W\) is
\(\Omega = \{HHHT, HHTH, HTHH, THHH\}\).

So, \(P(W) = \frac{4}{2^4} = \frac{1}{4} \hspace{8mm}\) and
\(P(W^{\complement}) = 1 - \frac{1}{4} = \frac{3}{4}\).

\begin{enumerate}
\def\labelenumi{\arabic{enumi})}
\tightlist
\item
  The expected value is: \[E[x] = \sum_{i=1}^{\infty} x_{i}p_{u}\]
\end{enumerate}

\(E[x] = \frac{1}{4}\cdot97 - \frac{3}{4}30\) = 1.75.

\begin{enumerate}
\def\labelenumi{\arabic{enumi})}
\setcounter{enumi}{1}
\tightlist
\item
  After 559, we'd expect to earn or lose \(559 \cdot E[x] =\) 978.25.
\end{enumerate}

\section{Question 12}\label{question-12}

Flip a coin 9 times. If you get 4 tails or less, I will pay you \$23.
Otherwise you pay me \$26.

Step 1. Find the expected value of the proposition. Round your answer to
two decimal places.

Step 2. If you played this game 994 times how much would you expect to
win or lose? (Losses must be entered as negative.)

\subsection{Solution}\label{solution-11}

Let \(W\) be the event that I win.

0 Tails: \(\binom{9}{0} = 1\).

1 Tail: \(\binom{9}{1} = 9\).

2 Tails: \(\binom{9}{2} = 36\).

3 Tails: \(\binom{9}{3} = 84\).

4 Tails: \(\binom{9}{4} = 126\).

\(\Omega = \{2^9\} = 512\).

\(P(W) = \frac{\sum W}{\Omega} = \frac{256}{512} = \frac{1}{2} = 0.5\).

\(P(W^{\complement}) = \frac{1}{2} = 0.5\).

\begin{enumerate}
\def\labelenumi{\arabic{enumi})}
\item
  \(E[x] = 23\cdot 0.5 - 26\cdot 0.5 =\) -1.5.
\item
  \(994\cdot E[x] = 994\cdot(-1.50) =\) -1491.
\end{enumerate}

\section{Question 13}\label{question-13}

The sensitivity and specificity of the polygraph has been a subject of
study and debate for years. A 2001 study of the use of polygraph for
screening purposes suggested that the probability of detecting a liar
was .59 (sensitivity) and that the probability of detecting a ``truth
teller'' was .90 (specificity). We estimate that about 20\% of
individuals selected for the screening polygraph will lie.

\begin{enumerate}
\def\labelenumi{\alph{enumi}.}
\item
  What is the probability that an individual is actually a liar given
  that the polygraph detected him/her as such? (Show me the table or the
  formulaic solution or both.)
\item
  What is the probability that an individual is actually a truth-teller
  given that the polygraph detected him/her as such? (Show me the table
  or the formulaic solution or both.)
\item
  What is the probability that a randomly selected individual is either
  a liar or was identified as a liar by the polygraph? Be sure to write
  the probability statement.
\end{enumerate}

\subsection{Solution}\label{solution-12}

Sensitivity = Correctly identify those with the attribute.

Specificity = Correctly identify those without the attribute.

Let \(L\) be the event that the selected individual is a liar.

\(P(L) = 0.20 \ \ \to P(L^{\complement}) = 0.80\).

Let \(DL\) be the event that a liar is detected, and \(DT\) be the event
that a truth-teller was detected.

\(P(DL|L) = 0.59 \ \ \to P(DL^{\complement}|L) = 0.41\).

\(P(DT|T) = 0.90 \ \ \to P(DT^{\complement}|T) = 0.10\).

Using Bayes' Theorem:

\begin{enumerate}
\def\labelenumi{\arabic{enumi})}
\tightlist
\item
  \(P(L|DL) = \frac{P(DL|L)P(L)}{P(DL|L)P(L) + P(DL|L^{\complement})P(L^{\complement})}\)
\end{enumerate}

\(P(L|DL) = \frac{0.59\cdot 0.20}{(0.59\cdot 0.20 + 0.1\cdot 0.80)} = 0.\overline{5959}\).

\begin{enumerate}
\def\labelenumi{\arabic{enumi})}
\setcounter{enumi}{1}
\tightlist
\item
  \(P(T|DT) = \frac{P(DT|T)P(T)}{P(DT|T)P(T) + P(DT|T^{\complement})P(T^{\complement})}\)
\end{enumerate}

\(P(T|DT) = \frac{0.90\cdot 0.80}{(0.90\cdot 0.80 + 0.41\cdot 0.20)} = 0.8978\).

\begin{enumerate}
\def\labelenumi{\arabic{enumi})}
\setcounter{enumi}{2}
\tightlist
\item
  Since the events \(L\) and \(DL\) are not disjoint, we use the formula
  \(P(L \cup DL) = P(L) + P(DL) - P(L \cap DL)\)
\end{enumerate}

To simplify, we can make use of DeMorgan's Law:

\(P(L \cup DL) = 1 - P(L \cup DL)^{\complement} = 1 - P(L^{\complement}\cap DL^{\complement})\)

\(\to 1 - (0.90 \cdot 0.80_ = 0.28\).


\end{document}
